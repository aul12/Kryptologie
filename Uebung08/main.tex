\documentclass[DIN, pagenumber=false, fontsize=11pt, parskip=half]{scrartcl}

\usepackage{amsmath}
\usepackage{amsfonts}
\usepackage{amssymb}
\usepackage{enumitem}
\usepackage[utf8]{inputenc} % this is needed for umlauts
\usepackage[ngerman]{babel}
\usepackage[T1]{fontenc} 
\usepackage{commath}
\usepackage{xcolor}
\usepackage{booktabs}
\usepackage{float}
\usepackage{tikz-timing}
\usepackage{tikz}
\usepackage{multirow}
\usepackage{colortbl}
\usepackage{xstring}
\usepackage{circuitikz}
\usepackage{listings} % needed for the inclusion of source code
\usepackage[final]{pdfpages}
\usepackage{subcaption}
\usepackage{import}
\usepackage{cleveref}
\usepackage{bm}
\usepackage{tcolorbox}

\usetikzlibrary{calc,shapes.multipart,chains,arrows}


% Default fixed font does not support bold face
\DeclareFixedFont{\ttb}{T1}{txtt}{bx}{n}{12} % for bold
\DeclareFixedFont{\ttm}{T1}{txtt}{m}{n}{12}  % for normal

\definecolor{deepblue}{rgb}{0,0,0.5}
\definecolor{deepred}{rgb}{0.6,0,0}
\definecolor{deepgreen}{rgb}{0,0.5,0}

\newcommand\pythonstyle{\lstset{
    language=Python,
    basicstyle=\scriptsize,
    otherkeywords={self},             % Add keywords here
    keywordstyle=\color{deepblue},
    emph={MyClass,__init__},          % Custom highlighting
    emphstyle=\color{deepred},    % Custom highlighting style
    stringstyle=\color{deepgreen},
    frame=tb,                         % Any extra options here
    showstringspaces=false,            % 
}}


% Python environment
\lstnewenvironment{python}[1][]
{
\pythonstyle
\lstset{#1}
}
{}

% Python for external files
\newcommand\pythonexternal[2][]{{
\pythonstyle
\lstinputlisting[#1]{#2}}}

% Python for inline
\newcommand\pythoninline[1]{{\pythonstyle\lstinline!#1!}}

\title{Übungsblatt 8}
\subtitle{Kryptologie: Algorithmen und Methoden}
\author{Paul Nykiel}

\begin{document}
    \maketitle
    \section{Aufgabe}
    Es sind $n, a, \tilde{x}, \tilde{y}$ bekannte, außerdem existiert eine
    Möglichkeit das El-Gamal-Problem zu lösen.

    Wähle ein beliebiges $\tilde{m}$ und löse das El-Gamal-Problem mit
    $n, a, \tilde{x}, \tilde{y}$ und $\tilde{m}$. Die Lösung ist ein $m$, so
    dass:
    \begin{equation}
        \tilde{m} = a^{x y} \cdot m \mod n
    \end{equation}
    bestimme nun das Inverse zu $m (\mod n)$ ($m^{-1}$), dies ist z.B. über den extggT effizient
    möglich, nun gilt:
    \begin{equation}
        \tilde{m} m^{-1} = a^{x y} \mod n = z
    \end{equation}
    damit ist auch das Diffie-Hellman-Problem gelöst, 

    \section{Aufgabe}
    Es gilt:
    \begin{eqnarray}
        h: &\{0, 1\}^* &\to \{0, 1\}^k \\
        g: &\{0, 1\}^* &\to \{0, 1\}^l \\
        f: &\{0, 1\}^l &\to \{0, 1\}^k
    \end{eqnarray}
    dabei sind $l, k$ beliebig $\in \mathbb{N}$. 
    Außerdem ist $h$ stark kollisionsresistent.

    Daraus folgt direkt, dass $f \circ g$ ebenfalls (stark) kollisionsresistent sein
    muss.

    Beweis per Widerspruch: Annahme $g$ ist nicht stark kollisionsresistent,
    das heißt es lassen sich effizient $a, b \in \{0, 1\}^*$ mit $a \neq b$ aber
    $g(a) = g(b)$ finden.

    Da $g(a) = g(b)$ gilt, gilt dann auch $f(g(a)) = f(g(b))$, bzw. $h(a) = h(b)$.
    Folglich lassen sich auch für $h$ effizient Kollisionen finden, d.h. 
    $h$ ist nicht stark kollisionsresistent. Dies ist ein Widerspruch zur Annahme.

    \section{Aufgabe}
    \begin{enumerate}[label=\alph*)]
        \item 
            Es gilt:
            \begin{equation}
                \tilde{m} = z^e \cdot m \mod n
            \end{equation}
            mit $z=3, e=11, m=9, n=119$, also mit Zahlenwerten:
            \begin{equation}
                \tilde{m} = 3^{11} \cdot 9 \mod 119 = 80
            \end{equation}
        \item
            Es gilt:
            \begin{equation}
                \tilde{u} = \tilde{m}^d \mod n
            \end{equation}
            zuerst muss $d$ berechnet werden, hierfür gilt:
            \begin{equation}
                e \cdot d \equiv 1 \mod \varphi(n)
            \end{equation} 

            da $n=17 \cdot 7$ gilt, folgt, dass $\varphi(n) = 16 \cdot 6 = 96$.
            
            Nun wird $d$ (das multiplikativ inverse) mithilfe des 
            $\text{extggT}(\varphi(n), e)$ berechnet:
            \begin{eqnarray}
                96 &=& 8 \cdot 11 + 8 \\
                11 &=& 1 \cdot 8 + 3 \\
                8 &=&  2 \cdot 3 + 2 \\
                3 &=& 1 \cdot 2 + 1 \\
                2 &=& 2 \cdot 1 + 0
            \end{eqnarray}
            rückwärts einsetzten:
            \begin{eqnarray}
                1 &=& 3 - 1 \cdot 2 \\ 
                  &=& 3 - 1 \cdot (8 - 2 \cdot 3) \\
                  &=& (-1) \cdot 8 + 3 \cdot 3\\
                  &=& (-1) \cdot 8 + 3 \cdot (11 - 8)\\
                  &=& 3 \cdot 11 + (-4) \cdot 8  \\
                  &=& 3 \cdot 11 + (-4) \cdot (96 - 8 \cdot 11)  \\
                  &=& 3 \cdot 11 + 4 \cdot (8 \cdot 11 - 96)  \\
                  &=& 35 \cdot 11 - 4 \cdot 96 
            \end{eqnarray}
            d.h. $d=35$.

            Damit lässt sich dann $\tilde{u}$ berechnen:
            \begin{equation}
                \tilde{u} = 80^{35} \mod 119 = 96
            \end{equation}

        \item 
            Es gilt:
            \begin{equation}
                u = \tilde{u} \cdot z^{-1} \mod n
            \end{equation}

            Zuerst $z^{-1}$ bestimmen, da $119+1=120=3 \cdot 40$ gilt, ist
            $z^{-1} = 40$.

            mit Zahlen gilt also:
            \begin{equation}
                u = 96 \cdot 40 \mod 119 = 32
            \end{equation}

            direktes Nachrrechen liefert das gleiche Ergebnis:
            \begin{equation}
                u = m^d \mod n = 9^{35} \mod 119 = 32
            \end{equation}
    \end{enumerate}

   %\section{Aufgabe}
\end{document}
