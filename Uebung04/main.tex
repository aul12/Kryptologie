\documentclass[DIN, pagenumber=false, fontsize=11pt, parskip=half]{scrartcl}

\usepackage{amsmath}
\usepackage{amsfonts}
\usepackage{amssymb}
\usepackage{enumitem}
\usepackage[utf8]{inputenc} % this is needed for umlauts
\usepackage[ngerman]{babel}
\usepackage[T1]{fontenc} 
\usepackage{commath}
\usepackage{xcolor}
\usepackage{booktabs}
\usepackage{float}
\usepackage{tikz-timing}
\usepackage{tikz}
\usepackage{multirow}
\usepackage{colortbl}
\usepackage{xstring}
\usepackage{circuitikz}
\usepackage{listings} % needed for the inclusion of source code
\usepackage[final]{pdfpages}
\usepackage{subcaption}
\usepackage{import}
\usepackage{cleveref}
\usepackage{bm}

\usetikzlibrary{calc,shapes.multipart,chains,arrows}


% Default fixed font does not support bold face
\DeclareFixedFont{\ttb}{T1}{txtt}{bx}{n}{12} % for bold
\DeclareFixedFont{\ttm}{T1}{txtt}{m}{n}{12}  % for normal

\definecolor{deepblue}{rgb}{0,0,0.5}
\definecolor{deepred}{rgb}{0.6,0,0}
\definecolor{deepgreen}{rgb}{0,0.5,0}

\newcommand\pythonstyle{\lstset{
    language=Python,
    basicstyle=\scriptsize,
    otherkeywords={self},             % Add keywords here
    keywordstyle=\color{deepblue},
    emph={MyClass,__init__},          % Custom highlighting
    emphstyle=\color{deepred},    % Custom highlighting style
    stringstyle=\color{deepgreen},
    frame=tb,                         % Any extra options here
    showstringspaces=false,            % 
}}


% Python environment
\lstnewenvironment{python}[1][]
{
\pythonstyle
\lstset{#1}
}
{}

% Python for external files
\newcommand\pythonexternal[2][]{{
\pythonstyle
\lstinputlisting[#1]{#2}}}

% Python for inline
\newcommand\pythoninline[1]{{\pythonstyle\lstinline!#1!}}

\title{Übungsblatt 4}
\subtitle{Kryptologie: Algorithmen und Methoden}
\author{Paul Nykiel}

\begin{document}
    \maketitle
    \section{Aufgabe}
    Siehe auch Appendix \ref{sec:app:1}.

    Für eine Gruppe $\mathbb{Z}_N$ muss gelten (hier bereits mit $\cdot$ als Operation):
    \begin{equation}
        \forall a \in \mathbb{Z}_N: \exists a^{-1} \in \mathbb{Z}_N: a \cdot a^{-1} = \varepsilon
    \end{equation}
    wobei $\varepsilon$ das neutrale Element bezüglich $\cdot$ ist.

    Für alle $\mathbb{Z}_N$ (mit $N>1$) gilt $\varepsilon=1$, da
    \begin{equation}
        a \cdot 1 = a\ \forall a \in \mathbb{Z}_N 
    \end{equation} 
    
    da aber gilt, dass:
    \begin{equation}
        \forall a \in \mathbb{Z}_N: a \cdot 0 = 0 \neq 1
    \end{equation} 
    hat $0$ kein multiplikativ Inverses in $\mathbb{Z}_N$ (mit $N>1$), somit ist
    kein $\mathbb{Z}_N$ keine Gruppe (bezüglich der Multiplikation), 
    und folglich $\mathbb{Z}_6$ ebenfalls nicht.

    \begin{table}[H]
    \centering
    \begin{tabular}{c|cccccc}
    \toprule
     & 0 & 1 & 2 & 3 & 4 & 5\\
    \midrule
    0 & 0 & 1 & 2 & 3 & 4 & 5\\
    1 & 1 & 2 & 3 & 4 & 5 & 0\\
    2 & 2 & 3 & 4 & 5 & 0 & 1\\
    3 & 3 & 4 & 5 & 0 & 1 & 2\\
    4 & 4 & 5 & 0 & 1 & 2 & 3\\
    5 & 5 & 0 & 1 & 2 & 3 & 4\\
    \bottomrule
    \end{tabular}
    \caption{Additionstabelle $\mathbb{Z}_6$}
    \end{table}

    \begin{table}[H]
    \centering
    \begin{tabular}{c|cccccc}
    \toprule
     & 0 & 1 & 2 & 3 & 4 & 5\\
    \midrule
    0 & 0 & 0 & 0 & 0 & 0 & 0\\
    1 & 0 & 1 & 2 & 3 & 4 & 5\\
    2 & 0 & 2 & 4 & 0 & 2 & 4\\
    3 & 0 & 3 & 0 & 3 & 0 & 3\\
    4 & 0 & 4 & 2 & 0 & 4 & 2\\
    5 & 0 & 5 & 4 & 3 & 2 & 1\\
    \bottomrule
    \end{tabular}
    \caption{Multiplikationstabelle $\mathbb{Z}_6$}
    \label{tab:mul6}
    \end{table}

    \begin{table}[H]
    \centering
    \begin{tabular}{c|ccccccc}
    \toprule
     & 0 & 1 & 2 & 3 & 4 & 5 & 6\\
    \midrule
    0 & 0 & 1 & 2 & 3 & 4 & 5 & 6\\
    1 & 1 & 2 & 3 & 4 & 5 & 6 & 0\\
    2 & 2 & 3 & 4 & 5 & 6 & 0 & 1\\
    3 & 3 & 4 & 5 & 6 & 0 & 1 & 2\\
    4 & 4 & 5 & 6 & 0 & 1 & 2 & 3\\
    5 & 5 & 6 & 0 & 1 & 2 & 3 & 4\\
    6 & 6 & 0 & 1 & 2 & 3 & 4 & 5\\
    \bottomrule
    \end{tabular}
    \caption{Additionstabelle $\mathbb{Z}_7$}
    \end{table}

    \begin{table}[H]
    \centering
    \begin{tabular}{c|ccccccc}
    \toprule
     & 0 & 1 & 2 & 3 & 4 & 5 & 6\\
    \midrule
    0 & 0 & 0 & 0 & 0 & 0 & 0 & 0\\
    1 & 0 & 1 & 2 & 3 & 4 & 5 & 6\\
    2 & 0 & 2 & 4 & 6 & 1 & 3 & 5\\
    3 & 0 & 3 & 6 & 2 & 5 & 1 & 4\\
    4 & 0 & 4 & 1 & 5 & 2 & 6 & 3\\
    5 & 0 & 5 & 3 & 1 & 6 & 4 & 2\\
    6 & 0 & 6 & 5 & 4 & 3 & 2 & 1\\
    \bottomrule
    \end{tabular}
    \caption{Multiplikationstabelle $\mathbb{Z}_7$}
    \end{table}

    \section{Aufgabe}
    \textbf{Hinweis:} hier wird nur gezeigt, dass ein Rechtsneutrales sowie -inverses
    Element existiert. Aus der Mathematik ist bekannt, dass daraus direkt gefolgt werden
    kann, dass das Element auch Linksneutral bzw. -invers ist.

    \subsection{Assoziativität}
    Für eine Gruppe muss gelten:
    \begin{equation}
        \forall a, b, c \in G:\ (a \circ b) \circ c = a \circ (b \circ c)
    \end{equation}
    Beweis:

    \subsection{Rechtsneutrales-Element}
    Für eine Gruppe muss gelten:
    \begin{equation}
        \exists \varepsilon \in G: \forall a \in G: a \cdot \varepsilon = a
    \end{equation}
    Beweis: Aus der Verknüpfungtabelle (erste Zeile) lässt sich ablesen, dass
    \begin{equation}
        \alpha \cdot a = a\ \forall a \in G
    \end{equation}
    folglich ist $\alpha$ das neutrale Element.

    \subsection{Rechtsinverses-Element}
    Für eine Gruppe muss gelten (hier bereits mit $\alpha$ als neutrales Element):
    \begin{equation}
        \forall a \in G \exists a^{-1} \in G: a \circ a^{-1} = \alpha
    \end{equation}
    Beweis: Aus der Verknüpfungstabelle (Hauptdiagonale) lässt sich ablesen, dass 
    \begin{equation}
        a \circ a = \alpha \forall a \in G
    \end{equation}
    folglich existiert für jedes $a \in G$ ein $a^{-1} = a$.

    \subsection{Kommutativität}
    Für eine abelsche Gruppe muss gelten:
    \begin{equation}
        \forall a, b \in G: a \circ b = b \circ a
    \end{equation}    
    Beweis: Aus der Verknüpfungstabelle (spiegeln über die Hauptdiagonal) lässt sich
    ablesen, dass
    \begin{equation}
        a \circ b = b \circ a \forall a, b \in G
    \end{equation}

    \section{Aufgabe}
    \begin{enumerate}[label=(\roman*)]
        \item
            $r$ und $s$ lassen sich in ihre Primfaktoren zerlegen:
            \begin{eqnarray}
                r &=& \prod_{i=1}^{k_r} {(p_{r,i})}^{e_{r,i}} \\
                s &=& \prod_{i=1}^{k_s} {(p_{s,i})}^{e_{s,i}}
            \end{eqnarray}
            Da $r$ und $s$ teilerfremd sind gilt:
            \begin{equation}
                p_{r,i} \neq p_{s,j}\ \forall i \in \{1,\ldots, k_r\}, j \in \{1,\ldots, k_s\}
            \end{equation}
            Definiere die $k_n = k_r + k_s$ Primfaktoren und zugehörigen Häufigkeiten von $n$:
            \begin{eqnarray}
                p_{n,i} &=&
                    \begin{cases}
                        p_{r,i} & \text{falls } i \leq k_r \\
                        p_{s,i-k_r} & \text{sonst}
                    \end{cases}\\
                e_{n,i} &=&
                    \begin{cases}
                        e_{r,i} & \text{falls } i \leq k_r \\
                        e_{s,i-k_r} & \text{sonst}
                    \end{cases}
            \end{eqnarray}

            daraus lässt sich die Primfaktorzerlegung von $n$ formulieren:
            \begin{equation}
                n = \prod_{i=1}^{k_n} {(p_{n,i})}^{e_{n,i}}
            \end{equation}

            Daraus folgt dann:
            \begin{eqnarray}
                \varphi(n) &=& \prod_{i=1}^{k_n} (p_{n,i} - 1) {(p_{n,i})}^{e_{n,i}} \\
                        &=& \prod_{i=1}^{k_r} (p_{n,i} - 1) {(p_{n,i})}^{e_{n,i}} \cdot
                            \prod_{i=k_r+1}^{k_n} (p_{n,i} - 1) {(p_{n,i})}^{e_{n,i}} \\
                        &=& \prod_{i=1}^{k_r} (p_{r,i} - 1) {(p_{r,i})}^{e_{r,i}} \cdot
                            \prod_{i=k_r+1}^{k_n} (p_{s,i-k_r} - 1) {(p_{s,i-k_r})}^{e_{s,i-k_r}} \\
                        &=& \prod_{i=1}^{k_r} (p_{r,i} - 1) {(p_{r,i})}^{e_{r,i}} \cdot
                            \prod_{i=1}^{k_n-k_r} (p_{s,i} - 1) {(p_{s,i})}^{e_{s,i}} \\
                        &=& \prod_{i=1}^{k_r} (p_{r,i} - 1) {(p_{r,i})}^{e_{r,i}} \cdot
                            \prod_{i=1}^{k_s} (p_{s,i} - 1) {(p_{s,i})}^{e_{s,i}} \\
                        &=& \varphi(r) \cdot \varphi(s)
            \end{eqnarray}
        \item
            $n$ und $m$ lassen sich in ihre Primfaktoren zerlegen:
            \begin{eqnarray}
                n &=& \prod_{i=1}^{k_n} {(p_{n,i})}^{e_{n,i}} \\
                m &=& \prod_{i=1}^{k_m} {(p_{m,i})}^{e_{m,i}}
            \end{eqnarray}

            Wir definieren die Menge der alleinigen Teiler von $n$, die Menge
            der alleinigen Teiler von $m$ und die Menge der Teiler beider Zahlen,
            jeweils mit zugehörigen Häufigkeiten:
            \begin{eqnarray}
                P_n &=& \left\{(p_{n,i}, e_{n,i}) | p_{n,i} \neq p_{m,j} \forall j \in \{1, \ldots k_m\} \right\} \\
                P_m &=& \left\{(p_{m,j}, e_{m,j}) | p_{m,j} \neq p_{n,i} \forall i \in \{1, \ldots k_n\} \right\} \\
                P_{n, m} &=& \left\{(p_{n,i}, e_{n,i}, e_{m_j}) | p_{n,i} = p_{m,j}, 
                    i \in \{1, \ldots k_n\}, j \in \{1, \ldots k_m\}\right\}
            \end{eqnarray}
            
            Der größte gemeinsame Teiler lässt sich dann als das Produkt
            der gemeinsamen Teiler definieren, dafür wird zuerst die Menge
            der Teiler $P_\text{ggT}$ mit zugehörigen (minimalen) Häufigkeiten definiert:
            \begin{equation}
                P_\text{ggT} = \left\{\left(p, \min(e_n, e_m)\right) | (p, e_n, e_m) \in P_{n,m} \right\}
            \end{equation}
            damit wird dann das ggT berechnet:
            \begin{equation}
                \text{ggT}(n, m) = \prod_{(p, e) \in P_\text{ggT}} p^e 
            \end{equation}

            Auch für das kleinste gemeinsame Vielfache lässt sich eine Menge analog
            zur obigen definieren. Diese enthält alle Teiler von $n$ und $m$ wobei
            bei mehrfachvorkommen immer nur der Überschuss behalten wird.
            \begin{equation}
                P_\text{kgV} = P_n \cup P_m \cup
                    \{(p, (e_n + e_m) - \min(e_n, e_m)) | (p, e_n, e_m) \in P_{n,m}\}
            \end{equation}
            da $a+b - \min(a,b) = \max(a,b)$ gilt, kann die Menge auch definiert werden
            als:
            \begin{equation}
                P_\text{kgV} = P_n \cup P_m \cup
                    \{(p, \max(e_n, e_m)) | (p, e_n, e_m) \in P_{n,m}\}
            \end{equation}

            analog zu oben wird daraus dann das kgV berechnet:
            \begin{equation}
                \text{kgV}(n, m) = \prod_{(p, e) \in P_\text{kgV}} p^e 
            \end{equation}

            Es ist zu beachten, dass die beiden Mengen zusammen genau alle Teiler von 
            $n\cdot m$, mit der Häufigkeiten (also quasi $P_{n,m}$) enthält.

            Mit diesen Definition nun zum eigentlichen Beweis:
            \begin{eqnarray}
                \varphi(n) \varphi(m) &=& \prod_{i=1}^{k_n} (p_{n,i}-1) {(p_{n,i})}^{e_{n,i}} \cdot \prod_{i=1}^{k_m} (p_{m,i}-1) {(p_{m,i})}^{e_{m,i}} \\
                    &=& \prod_{(p, e) \in P_n} (p-1) p^e \cdot \prod_{(p,e) \in P_m} (p-1) p^e \\
                    && \cdot \prod_{(p, e_n, e_m) \in P_{n, m}} {(p-1)}^2 p^{e_n + e_m} \\
                    &=& \prod_{(p, e) \in P_n} (p-1) p^e \cdot \prod_{(p,e) \in P_m} (p-1) p^e \\
                    && \cdot \prod_{(p, e_n, e_m) \in P_{n, m}} {(p-1)}^2 p^{\max(e_n, e_m) + \min(e_n, e_m)} \\
                    &=& \prod_{(p, e) \in P_n} (p-1) p^e \cdot \prod_{(p,e) \in P_m} (p-1) p^e \\
                    && \cdot \prod_{(p, e_n, e_m) \in P_{n, m}} (p-1) p^{\max(e_n, e_m)} (p-1) p^{\min(e_n, e_m)} \\
                    &=& \prod_{(p, e) \in P_n} (p-1) p^e \cdot \prod_{(p,e) \in P_m} (p-1) p^e \\
                    && \cdot \prod_{(p, e_n, e_m) \in P_{n, m}} (p-1) p^{\max(e_n, e_m)} \cdot \prod_{(p,e) \in P_\text{ggT}}  (p-1) p^e \\
                    &=& \prod_{(p,e) \in P_\text{kgV}} (p-1) p^e \cdot \prod_{(p,e) \in P_\text{ggT}}  (p-1) p^e \\
                    &=& \varphi(\text{kgV}(n, m)) \cdot \varphi(\text{ggT}(n, m)) \\
                    &=& \varphi(\text{ggT}(n, m)) \cdot \varphi(\text{kgV}(n, m))
            \end{eqnarray}
                
    \end{enumerate}

    \appendix
    \section{Aufgabe 1} \label{sec:app:1}
    \pythonexternal{aufgabe1.py}    
\end{document}
