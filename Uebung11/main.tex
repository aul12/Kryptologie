\documentclass[DIN, pagenumber=false, fontsize=11pt, parskip=half]{scrartcl}

\usepackage{amsmath}
\usepackage{amsfonts}
\usepackage{amssymb}
\usepackage{enumitem}
\usepackage[utf8]{inputenc} % this is needed for umlauts
\usepackage[ngerman]{babel}
\usepackage[T1]{fontenc} 
\usepackage{commath}
\usepackage{xcolor}
\usepackage{booktabs}
\usepackage{float}
\usepackage{tikz-timing}
\usepackage{tikz}
\usepackage{multirow}
\usepackage{colortbl}
\usepackage{xstring}
\usepackage{circuitikz}
\usepackage{listings} % needed for the inclusion of source code
\usepackage[final]{pdfpages}
\usepackage{subcaption}
\usepackage{import}
\usepackage{cleveref}
\usepackage{bm}
\usepackage{tcolorbox}
\usepackage{mathtools}

\usetikzlibrary{calc,shapes.multipart,chains,arrows}

\DeclarePairedDelimiter{\ceil}{\lceil}{\rceil}


% Default fixed font does not support bold face
\DeclareFixedFont{\ttb}{T1}{txtt}{bx}{n}{12} % for bold
\DeclareFixedFont{\ttm}{T1}{txtt}{m}{n}{12}  % for normal

\definecolor{deepblue}{rgb}{0,0,0.5}
\definecolor{deepred}{rgb}{0.6,0,0}
\definecolor{deepgreen}{rgb}{0,0.5,0}

\newcommand\pythonstyle{\lstset{
    language=Python,
    basicstyle=\scriptsize,
    otherkeywords={self},             % Add keywords here
    keywordstyle=\color{deepblue},
    emph={MyClass,__init__},          % Custom highlighting
    emphstyle=\color{deepred},    % Custom highlighting style
    stringstyle=\color{deepgreen},
    frame=tb,                         % Any extra options here
    showstringspaces=false,            % 
}}


% Python environment
\lstnewenvironment{python}[1][]
{
\pythonstyle
\lstset{#1}
}
{}

% Python for external files
\newcommand\pythonexternal[2][]{{
\pythonstyle
\lstinputlisting[#1]{#2}}}

% Python for inline
\newcommand\pythoninline[1]{{\pythonstyle\lstinline!#1!}}

\title{Übungsblatt 11}
\subtitle{Kryptologie: Algorithmen und Methoden}
\author{Paul Nykiel}

\begin{document}
    \maketitle
    \section{Aufgabe}
    Zu Begin: $a=2, B=1, n=24823$.
    \begin{enumerate}
        \item
            \begin{eqnarray}
                B &=& B+1 = 2 \\
                a &=& a^B \mod n = 4 \\
                \text{ggT}(a-1, n) &=& 1 
            \end{eqnarray}
        \item
            \begin{eqnarray}
                B &=& B+1 = 3 \\
                a &=& a^B \mod n = 64 \\
                \text{ggT}(a-1, n) &=& 1 
            \end{eqnarray}
        \item
            \begin{eqnarray}
                B &=& B+1 = 4 \\
                a &=& a^B \mod n = 21691 \\
                \text{ggT}(a-1, n) &=& 1 
            \end{eqnarray}
        \item
            \begin{eqnarray}
                B &=& B+1 = 5 \\
                a &=& a^B \mod n = 2893 \\
                \text{ggT}(a-1, n) &=& 241 
            \end{eqnarray}
    \end{enumerate}
    Einer der Teiler von $n$ ist $241$. 
    Damit ergibt sich $\tilde{n} = \frac{n}{241} = 103$, welches ebenfalls eine
    Primzahl ist, folglich sind die (einzigen) beiden Primfaktoren $103$ und $241$.

    \section{Aufgabe}
    Es gilt: $x = y = 20, n=1219$.
    \begin{enumerate}
        \item 
            \begin{eqnarray}
                x &=& x^2 + 1 \mod 1219 = 401 \\
                \tilde{y} &=& y^2 + 1 \mod 1219 = 401 \\
                y &=& \tilde{y}^2 + 1 \mod 1219 = 1113 \\
                g &=& \text{ggT}(x-y, n) = 1
            \end{eqnarray}
        \item 
            \begin{eqnarray}
                x &=& x^2 + 1 \mod 1219 = 1113 \\
                \tilde{y} &=& y^2 + 1 \mod 1219 = 266 \\
                y &=& \tilde{y}^2 + 1 \mod 1219 = 55 \\
                g &=& \text{ggT}(x-y, n) = 23
            \end{eqnarray}
    \end{enumerate} 
    Einer der Teiler von $n$ ist $23$. 
    Damit ergibt sich $\tilde{n} = \frac{n}{23} = 53$ welches ebenfalls eine Primzahl ist,
    folglich sind die (einzigen) beiden Primfaktoren $23$ und $53$.

    \section{Aufgabe}
    \textbf{Anmerkung:} im Skript wird das neue $z$ durch $z + 2 \cdot x - 1$ berechnet, es muss jedoch durch
    $z+ 2 \cdot x + 1$ berechnet werden, da:
    \begin{eqnarray}
        z &=& x^2 -n \\
        \tilde{z} &=& {(x+1)}^2 - n \\
        &=& x^2 + 2 \cdot x + 1 - n \\
        &=& x^2 - n + 2 \cdot x + 1 \\
        &=& z + 2 \cdot x + 1
    \end{eqnarray}
    Mit dem Algorithmus von Fermat zur Faktorisierung wird keine Lösung gefunden, siehe
    \ref{sec:app:3}.

    \section{Aufgabe}
    \begin{enumerate}[label=(\alph*)]
        \item Aus den fünf möglichen Werten für $x$ können jeweils die beiden
            möglichen $y$ Werte bestimmt werden. Siehe hierfür \ref{sec:app:4}.
            
            Die Punkte sind: $(0, 0), (2, 0), (3, 0)$. Für $x=1$ und $x=4$ existieren
            keine Lösungen.
            
    \end{enumerate}

    \appendix
    \section{Aufgabe 3} \label{sec:app:3}
    \pythonexternal{aufgabe3.py}

    \section{Aufgabe 4} \label{sec:app:4}
    \pythonexternal{aufgabe4.py}
\end{document}
