\documentclass[DIN, pagenumber=false, fontsize=11pt, parskip=half]{scrartcl}

\usepackage{amsmath}
\usepackage{amsfonts}
\usepackage{amssymb}
\usepackage{enumitem}
\usepackage[utf8]{inputenc} % this is needed for umlauts
\usepackage[ngerman]{babel}
\usepackage[T1]{fontenc} 
\usepackage{commath}
\usepackage{xcolor}
\usepackage{booktabs}
\usepackage{float}
\usepackage{tikz-timing}
\usepackage{tikz}
\usepackage{multirow}
\usepackage{colortbl}
\usepackage{xstring}
\usepackage{circuitikz}
\usepackage{listings} % needed for the inclusion of source code
\usepackage[final]{pdfpages}
\usepackage{subcaption}
\usepackage{import}
\usepackage{cleveref}
\usepackage{bm}

\usetikzlibrary{calc,shapes.multipart,chains,arrows}


% Default fixed font does not support bold face
\DeclareFixedFont{\ttb}{T1}{txtt}{bx}{n}{12} % for bold
\DeclareFixedFont{\ttm}{T1}{txtt}{m}{n}{12}  % for normal

\definecolor{deepblue}{rgb}{0,0,0.5}
\definecolor{deepred}{rgb}{0.6,0,0}
\definecolor{deepgreen}{rgb}{0,0.5,0}

\newcommand\pythonstyle{\lstset{
    language=Python,
    basicstyle=\scriptsize,
    otherkeywords={self},             % Add keywords here
    keywordstyle=\color{deepblue},
    emph={MyClass,__init__},          % Custom highlighting
    emphstyle=\color{deepred},    % Custom highlighting style
    stringstyle=\color{deepgreen},
    frame=tb,                         % Any extra options here
    showstringspaces=false,            % 
}}


% Python environment
\lstnewenvironment{python}[1][]
{
\pythonstyle
\lstset{#1}
}
{}

% Python for external files
\newcommand\pythonexternal[2][]{{
\pythonstyle
\lstinputlisting[#1]{#2}}}

% Python for inline
\newcommand\pythoninline[1]{{\pythonstyle\lstinline!#1!}}

\newcommand{\Prb}[1]{P(\text{#1})}
\newcommand{\CPr}[2]{P(\text{#1}|\text{#2})}
\DeclareMathOperator*{\argmax}{arg\,max}
\DeclareMathOperator*{\argmin}{arg\,min}
\DeclareMathOperator{\rank}{rank}
\newcommand{\R}[0]{\mathbb{R}}

\title{Übungsblatt 1}
\subtitle{Kryptologie: Algorithmen und Methoden}
\author{Paul Nykiel}

\begin{document}
    \maketitle
    
    \section{Aufgabe}
    Vermutlich korrespondiert der verschlüsselte Text zum deutschen Wort \glqq{}Was\grqq{},
    durch ausprobieren aller 26 Schlüssel kann die Wahrscheinlichkeit für jede
    Buchstabenkombination bestimmt werden (unter den Annahme statistischer Unabhängigkeit zwischen den Buchstaben). 
    Hierbei ist "Was" die viert wahrscheinlichste Möglichkeit (mit einer Wahrscheinlichkeit von ca. 5\%) und die einzige die einem deutschen Wort entspricht (siehe Appendix \ref{sec:app:1} für das Skript zum berechnen der Wahrscheinlichkeiten).

    Bei einem einzelnen Buchstaben kann offensichtlich noch keine sinnvolle Ausage getroffen werden, der wahrscheinlichste Schlüssel ist hier immer jener, der den Buchstaben \glqq{}E\grqq{} erzeugt. 
    Auch bei zwei Buchstaben ist die Entschlüsselung meist nicht eindeutig, so haben schon die beiden im Skript als häufig gelisteten Bigramme \glqq{}CH\grqq{} und \glqq{}IN\grqq{} den gleichen Abstand zwischen den Buchstaben (21) also ist auch hier eine eindeutige dechiffrierung Unwahrscheinlich.
    Bei drei Buchstaben kann meistens davon ausgegangen werden das es nur noch sehr wenige Doppelungen gibt, oder sogar gar keine wie im obigen Beispiel.
    Folglich steht der Schlüssel ab drei bis vier Buchstaben mit hoher Wahrscheinlichkeit fest.

    \section{Aufgabe}
    \begin{enumerate}[label=(\roman*)]
        \item $k_1$ muss Teilerfremd zu $26$ sein, also muss gelten $\text{ggT}(k_1, 26) = 1$. 
            12 Zahlen erfüllen diese Bedingung. Für $k_2$ gibt es keine Einschränkungen, folglich ist die Mächtigkeit der Menge aller möglichen $k_2$ $26$.

            $k_1$ und $k_2$ können beliebig kombiniert werden, folglich gibt es maximal $12 \cdot 26 = 312$ mögliche Schlüssel.
        \item Nur genau wenn die Anforderung der Teilerfremdheit 
            (also $\text{ggT}(k_1, 26) = 1$) erfüllt ist, dann ist eine Eindeutige
            Dechiffrierung möglich.

            \textbf{Begründung:} Annahme 
            $\text{ggT}(k_1, 26) = n,\ n \in \mathbb{N}, n > 1$, dann gilt:
            
            \begin{eqnarray*}
                E\left((k_1, k_2), 0\right) &=& k_1 \cdot 0 +k_2 \mod 26 \\
                    &=& k_2 \mod 26 \\
                    &\stackrel{k_2 < 26}{=}& k_2 
            \end{eqnarray*}
            Sowie:
            \begin{eqnarray*}
                E\left((k_1, k_2), \frac{26}{n}\right) &=& k_1 \cdot \frac{26}{n} + k_2 \mod 26 \\
                    &=& 26 \cdot \frac{k_1}{n} + k_2 \mod 26  \\
                    &\stackrel{\frac{k_1}{n} \in \mathbb{Z}_{26}}{=}& k_2 \mod 26 \\
                    &\stackrel{k_2 < 26}{=}& k_2 
            \end{eqnarray*}

            Es gilt also das $E\left((k_1, k_2), 0\right) = E\left((k_1, k_2), \frac{26}{n}\right)$ 
            obwohl $0 \in \mathbb{Z}_{26}, \frac{26}{n} \in \mathbb{Z}_{26}$ und $0 \neq \frac{26}{n}$. Daraus folgt das wenn $k_1$ nicht Teilerfremd zu 26 ist eine eindeutige Chiffrierung nicht für alle Zeichen möglich ist.
        \item Es gilt $D(k, x) = k_1^{-1} \cdot (x - k_2) \mod 26$.
            Außerdem gilt $3^{-1} = 9$, da:
            \begin{eqnarray*}
                3 \cdot 3^{-1} &=& 3 \cdot 9 \mod 26 \\
                    &=& 27 \mod 26 \\
                    &=& 1
            \end{eqnarray*}

            Einsetzten ergibt:
            \begin{eqnarray*}
                D(k, E(k, x)) &=& 3^{-1} \cdot (E(k, x) - 5) \mod 26\\
                    &=& 3^{-1} \cdot \left((3 \cdot x + 5) - 5 \right) \mod 26 \\
                    &=& 9 \cdot (3 \cdot x) \mod 26 \\
                    &=& (9 \cdot 3) \cdot x \mod 26 \\
                    &=& 26 \cdot x + x \mod 26 \\
                    &=& x \mod 26 \\
                    &=& x
            \end{eqnarray*}
    \end{enumerate}

    \section{Aufgabe}
    Es kommen das Triple \glqq{}oqg\grqq{} sowie die Kombination \glqq{}qgyxo\grqq{} mehrfach vor.
    Der Abstand zwischen den Vorkommen der 5-fach Kombination beträgt 20 Zeichen, zwischen den vorkommen
    des Triples 28 Zeichen. Die Wahrscheinliche Schlüssellänge ist also $\text{ggt}(20, 28) = 4$.
    
    Schlüsselhypothesen wurden aufgrund der Buchstabenhäufigkeiten bestimmt (siehe Appendix \ref{sec:app:31}).

    \section{Aufgabe}
    Text: \glqq{}ICHTESTEKRYPTOGRAPHIE\grqq{}

    Schlüssel: \glqq{}KRYPTO\grqq{}

    \begin{table}[H]
        \centering
        \begin{tabular}{|c|c|c|c|c|}
            \toprule
            K & R & Y & P & T \\
            \midrule
            O & A & B & C & D \\
            \midrule
            E & F & G & H & I \\
            \midrule
            L & M & N & Q & S \\
            \midrule
            U & V & W & X & Z \\
            \bottomrule
        \end{tabular}
        \caption{Tabelle für die Playfair Verschlüsselung}
    \end{table}
    
    \begin{table}[H]
        \centering
        \begin{tabular}{cc}
            \toprule
            Bigramm & Verschlüsselt \\
            \midrule
            IC & DH \\
            HT & IP \\
            ES & LI \\
            TE & KI \\
            KR & RY \\
            YP & PT \\ 
            TO & KD \\
            GR & YF \\
            AP & CR \\
            HI & IE \\
            EX & UH \\
            \bottomrule
        \end{tabular}
        \caption{Substitutionstabelle für alle vorkommenden Bigramme}
    \end{table}

    Verschlüsselter Text: \glqq{}DHIPLIKIRYPTKDYFCRIEUH\grqq{}

    \section{Aufgabe}
    \begin{enumerate}[label=\alph*)]
        \item 
            Siehe auch Appendix \ref{sec:app:5}.
            \begin{enumerate}[label=\alph*)]
                \item $ $
                    \begin{table}[H]
                        \centering
                        \begin{tabular}{ccc}
                            \toprule
                            Text & Schlüssel & Verschlüsselt \\
                            \midrule
                            t & p & i \\
                            r & l & c \\
                            e & a & e \\
                            f & n & s \\
                            f & t & y \\
                            e & r & v \\
                            n & e & r \\
                            m & f & r \\
                            i & f & n \\
                            t & e & x \\
                            t & n & g \\
                            w & m & i \\
                            o & i & w \\
                            c & t & v \\
                            h & t & a \\
                            n & w & j \\
                            e & o & s \\
                            u & c & w \\
                            n & h & u \\
                            u & n & h \\
                            h & e & l \\
                            r & u & l \\
                            b & n & o \\
                            a & u & u \\
                            h & h & o \\
                            n & r & e \\
                            h & b & i \\
                            o & a & o \\
                            f & h & m \\
                            f & n & s \\
                            r & h & y \\
                            i & o & w \\
                            e & f & j \\
                            d & f & i \\
                            r & r & i \\
                            i & i & q \\
                            c & e & g \\
                            h & d & k \\
                            s & r & j \\
                            t & i & b \\
                            r & c & t \\
                            a & h & h \\
                            s & s & k \\
                            s & t & l \\
                            e & r & v \\
                            \bottomrule
                        \end{tabular}
                    \end{table}
            \end{enumerate}

            Verschlüsselt: \glqq{}icesyvrrnxgiwvajswuhllouoeiomsywjiiqgkjbthklv\grqq{}
        \item TODO
    \end{enumerate}

    \appendix
    \section{Aufgabe 1} \label{sec:app:1}
    Siehe auch Appendix \ref{sec:app:caesar}.
    \pythonexternal{aufgabe1.py}

    \section{Aufgabe 3 - Schlüsselhypthosesn} \label{sec:app:31}
    Siehe auch Appendix \ref{sec:app:caesar}.
    \pythonexternal{aufgabe3_1.py}

    \section{Aufgabe 3 - Dekodieren} \label{sec:app:32}
    Siehe auch Appendix \ref{sec:app:caesar}.
    \pythonexternal{aufgabe3_2.py}

    \section{Aufgabe 5} \label{sec:app:5}
    Siehe auch Appendix \ref{sec:app:caesar}.
    \pythonexternal{aufgabe5.py}

    \section{caesar.py} \label{sec:app:caesar}
    \pythonexternal{caesar.py}
\end{document}
