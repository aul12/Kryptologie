\documentclass[DIN, pagenumber=false, fontsize=11pt, parskip=half]{scrartcl}

\usepackage{amsmath}
\usepackage{amsfonts}
\usepackage{amssymb}
\usepackage{enumitem}
\usepackage[utf8]{inputenc} % this is needed for umlauts
\usepackage[ngerman]{babel}
\usepackage[T1]{fontenc} 
\usepackage{commath}
\usepackage{xcolor}
\usepackage{booktabs}
\usepackage{float}
\usepackage{tikz-timing}
\usepackage{tikz}
\usepackage{multirow}
\usepackage{colortbl}
\usepackage{xstring}
\usepackage{circuitikz}
\usepackage{listings} % needed for the inclusion of source code
\usepackage[final]{pdfpages}
\usepackage{subcaption}
\usepackage{import}
\usepackage{cleveref}
\usepackage{bm}
\usepackage{tcolorbox}

\usetikzlibrary{calc,shapes.multipart,chains,arrows}


% Default fixed font does not support bold face
\DeclareFixedFont{\ttb}{T1}{txtt}{bx}{n}{12} % for bold
\DeclareFixedFont{\ttm}{T1}{txtt}{m}{n}{12}  % for normal

\definecolor{deepblue}{rgb}{0,0,0.5}
\definecolor{deepred}{rgb}{0.6,0,0}
\definecolor{deepgreen}{rgb}{0,0.5,0}

\newcommand\pythonstyle{\lstset{
    language=Python,
    basicstyle=\scriptsize,
    otherkeywords={self},             % Add keywords here
    keywordstyle=\color{deepblue},
    emph={MyClass,__init__},          % Custom highlighting
    emphstyle=\color{deepred},    % Custom highlighting style
    stringstyle=\color{deepgreen},
    frame=tb,                         % Any extra options here
    showstringspaces=false,            % 
}}


% Python environment
\lstnewenvironment{python}[1][]
{
\pythonstyle
\lstset{#1}
}
{}

% Python for external files
\newcommand\pythonexternal[2][]{{
\pythonstyle
\lstinputlisting[#1]{#2}}}

% Python for inline
\newcommand\pythoninline[1]{{\pythonstyle\lstinline!#1!}}

\title{Übungsblatt 10}
\subtitle{Kryptologie: Algorithmen und Methoden}
\author{Paul Nykiel}

\begin{document}
    \maketitle
    \section{Aufgabe}
    Es werden drei Geheimnisse benötigt, d.h. $k=3$, das gesuchte Polynom hat Grad $k-1=2$,
    ist also der Form:
    \begin{equation}
        A(t) = z_0 + z_1 \cdot t + z_2 \cdot t^2 \mod 13
    \end{equation}
    Außerdem sind drei $(x, y)$ Tuple gegeben. Damit gilt dann:
    \begin{eqnarray}
        7 &=& z_0 + z_1 + z_2 \mod 13 \\
        5 &=& 5 \cdot z_0 + 25 \cdot z_1 + 125 \cdot z_2 \mod 13 \\
        \Leftrightarrow 5 &=& 5 \cdot z_0 + 12 \cdot z_1 + 8 \cdot z_2 \mod 13 \\
        3 &=& 12 \cdot z_0 + 144 \cdot z_1 + 1728 \cdot z_2 \mod 13 \\
        \Leftrightarrow 3 &=& 12 \cdot z_0 + z_1 + 12 \cdot z_2 \mod 13 
    \end{eqnarray}
    d.h. es muss nur ein lineares Gleichungssystem gelöst werden, mit der
    Methode von Gauß:
    \begin{eqnarray}
        \begin{pmatrix}
            1 & 1 & 1 \\
            5 & 12 & 8 \\
            12 & 1 & 12
        \end{pmatrix}
        \begin{pmatrix}
            z_0 \\ z_1 \\ z_2
        \end{pmatrix}
        &=&
        \begin{pmatrix}
            7 \\ 5 \\ 3
        \end{pmatrix} \mod 13 \\
        \Leftrightarrow
        \begin{pmatrix}
            1 & 1 & 1 \\
            5 & 12 & 8 \\
            0 & 2 & 0
        \end{pmatrix}
        \begin{pmatrix}
            z_0 \\ z_1 \\ z_2
        \end{pmatrix}
        &=&
        \begin{pmatrix}
            7 \\ 5 \\ 10
        \end{pmatrix} \mod 13 \\
        \Leftrightarrow
        \begin{pmatrix}
            1 & 1 & 1 \\
            0 & 7 & 3 \\
            0 & 2 & 0
        \end{pmatrix}
        \begin{pmatrix}
            z_0 \\ z_1 \\ z_2
        \end{pmatrix}
        &=&
        \begin{pmatrix}
            7 \\ 9 \\ 10
        \end{pmatrix} \mod 13
    \end{eqnarray}
    Also gilt:
    \begin{equation}
        2 \cdot z_1 = 10 \mod 13
    \end{equation}
    d.h. $z_2 = 5$. Die zweite Zeile liefert dann:
    \begin{eqnarray}
        7 \cdot 5 + 3 \cdot z_2 &=& 5 \mod 13 \\
        35 + 3 \cdot z_2 &=& 5 \mod 13 \\
        3 \cdot z_2 &=& 4 \mod 13 \\
        z_2 &=& 4 \cdot 3^{-1} \mod 13 \\
        z_2 &=& 4 \cdot 9 \mod 13 \\
        z_2 &=& 10 \mod 13 \\
    \end{eqnarray}
    damit kann dann die erste Zeile genutzt werden:
    \begin{eqnarray}
        z_0 + 5 + 10 &=& 7 \mod 13 \\
        z_0 &=& 7 - 15 \mod 13 \\
        z_0 &=& 5 \mod 13
    \end{eqnarray}
    also gilt: $z_0 = 5, z_1 = 5, z_2 = 10$.

    Das Geheimnis ist dann gegeben durch:
    \begin{equation}
        A(0) = z_0 = 5
    \end{equation}
\end{document}
