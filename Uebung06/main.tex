\documentclass[DIN, pagenumber=false, fontsize=11pt, parskip=half]{scrartcl}

\usepackage{amsmath}
\usepackage{amsfonts}
\usepackage{amssymb}
\usepackage{enumitem}
\usepackage[utf8]{inputenc} % this is needed for umlauts
\usepackage[ngerman]{babel}
\usepackage[T1]{fontenc} 
\usepackage{commath}
\usepackage{xcolor}
\usepackage{booktabs}
\usepackage{float}
\usepackage{tikz-timing}
\usepackage{tikz}
\usepackage{multirow}
\usepackage{colortbl}
\usepackage{xstring}
\usepackage{circuitikz}
\usepackage{listings} % needed for the inclusion of source code
\usepackage[final]{pdfpages}
\usepackage{subcaption}
\usepackage{import}
\usepackage{cleveref}
\usepackage{bm}
\usepackage{tcolorbox}

\usetikzlibrary{calc,shapes.multipart,chains,arrows}


% Default fixed font does not support bold face
\DeclareFixedFont{\ttb}{T1}{txtt}{bx}{n}{12} % for bold
\DeclareFixedFont{\ttm}{T1}{txtt}{m}{n}{12}  % for normal

\definecolor{deepblue}{rgb}{0,0,0.5}
\definecolor{deepred}{rgb}{0.6,0,0}
\definecolor{deepgreen}{rgb}{0,0.5,0}

\newcommand\pythonstyle{\lstset{
    language=Python,
    basicstyle=\scriptsize,
    otherkeywords={self},             % Add keywords here
    keywordstyle=\color{deepblue},
    emph={MyClass,__init__},          % Custom highlighting
    emphstyle=\color{deepred},    % Custom highlighting style
    stringstyle=\color{deepgreen},
    frame=tb,                         % Any extra options here
    showstringspaces=false,            % 
}}


% Python environment
\lstnewenvironment{python}[1][]
{
\pythonstyle
\lstset{#1}
}
{}

% Python for external files
\newcommand\pythonexternal[2][]{{
\pythonstyle
\lstinputlisting[#1]{#2}}}

% Python for inline
\newcommand\pythoninline[1]{{\pythonstyle\lstinline!#1!}}

\title{Übungsblatt 6}
\subtitle{Kryptologie: Algorithmen und Methoden}
\author{Paul Nykiel}

\begin{document}
    \maketitle
    \section{Aufgabe}
    Die vorgestellte Lösung implementiert ein Commitment Schema unter der Ausnutzung
    der nicht effizienten Lösbarkeit des diskreten-Logarithmus.

    Beide Partner wählen ihren jeweiligen Schlüssel ($x$ bzw. $y$), außerdem wählen 
    beide einen eigene Primzahl ($c$ bzw. $d$). Aus $x$ bzw. $y$ können Alice und Bob
    bereits $\tilde{x}$ bzw. $\tilde{y}$ berechnen.

    Im ersten Schritt wird $\tilde{x}$ und $\tilde{y}$ verschlüsselt an den jeweilig 
    anderen übertragen, hierfür berechnet Alice $h=\tilde{x}^c$ und Bob $j=\tilde{y}^d$.
    Alice überträgt dann $h$ und Bob und Bob überträgt $j$ an Alice.
    Die Reihenfolge hiervon ist egal.

    Da der diskrete-Logarithmus nicht effizient zu berechnen ist können weder Alice 
    noch Bob aus der Empfangenen Zahl $\tilde{y}$ bzw. $\tilde{x}$ berechnen.
    Es ist aber für Bob und Alice auch nicht (effizient) möglich eine andere Kombination
    aus $\tilde{x}$ und $c$ bzw. $\tilde{y}$ und $d$ zu wählen die die selbe Nachricht
    erzeugt.
         
    Nun wird quasi ein normaler DH-Schlüsseltausch durchgeführt: Alice sendet 
    $\tilde{x}$ und $c^{-1}$, Bob sendet $\tilde{y}$ und $d^{-1}$. Aus $\tilde{x}$ und
    $y$ kann Bob $z$ berechnen, Alice verwendet dafür $\tilde{y}$ und $x$.
    Zur Verifikation, dass die Zahl $\tilde{x}$ bzw. $\tilde{y}$ die ursprünglich
    gewählte ist, können Alice und Bob die ursprüngliche Zahl entschlüsseln, also
    $\tilde{x}^{c^{c^{-1}}} = \tilde{x}$ bzw. $\tilde{y}^{d^{d^{-1}}} = \tilde{y}$ berechnen.

    \section{Aufgabe}
    Die Wahrscheinlichkeit, dass das korrekte Ergebniss nach $t$ vorliegt ist gegeben durch:
    \begin{equation}
        p_\text{Korrekt} = 1 - \frac{1}{2} {\left(1 - 4 \varepsilon^2 \right)}^{\frac{t}{2}}
    \end{equation}
    
    nach $t$ auflösen:
    \begin{eqnarray}
        p_\text{Korrekt} &=& 1 - \frac{1}{2} {\left(1 - 4 \varepsilon^2 \right)}^{\frac{t}{2}} \\
        2 \cdot (1 - p_\text{Korrekt}) &=& {\left(1 - 4 \varepsilon^2 \right)}^{\frac{t}{2}} \\
        \log\left(2 \cdot (1 - p_\text{Korrekt})\right) &=& \frac{t}{2} \log \left(1 - 4 \varepsilon^2 \right) \\
        2 \cdot \frac{\log\left(2 \cdot (1 - p_\text{Korrekt})\right)}{\log \left(1 - 4 \varepsilon^2 \right)} &=& t  \\
    \end{eqnarray}

    Für $\varepsilon = \frac{1}{3}$ und $p_\text{Korrekt}=1-0.001$ gilt dann:
    \begin{equation}
        t = 2 \cdot \frac{\log\left(2 \cdot (1 - p_\text{Korrekt})\right)}{\log \left(1 - 4 \varepsilon^2 \right)} = 21.14
    \end{equation}
    Der Algorithmus muss mindestens $22$ mal ausgeführt werden um die Fehlerwahrscheinlichkeit auf unter $0.001$ zu drücken.

    \section{Aufgabe}
    Miller-Rabin-Primzahltest für $n=65$.

    $n-1$ in Binärdarstellung:
    \begin{equation}
        n-1 = 64 = {1000000}_2 = (b_6, b_5, b_4, b_3, b_2, b_1, b_0)
    \end{equation}
    außerdem gilt $k=6$, $d=1$ und $dd=1$.

    \subsection{$a=8$}
    \begin{enumerate}
        \item Schleife: $i=k=6$
        \item $d = d \cdot d \mod n = 1 \cdot 1 \mod 65 = 1$
        \item Erstes If: $d=1$ und $dd=1$, also kein return
        \item Zweites If: $b_i = b_6 = 1$, also $d = d \cdot a \mod n = 1 \cdot 8 \mod 65 = 8$
        \item $dd=d=8$
        \item Zweiter Schleifendurchlauf: $i=5$
        \item $d = 64 \mod 65 = 64$  
        \item Erstes If: $d \neq 1$, also kein return
        \item Zweites If: $b_5 \neq 1$, also $d$ nicht verändern
        \item $dd = d = 64$
        \item Dritter Schleifendurchlauf: $i=4$
        \item $d = 64^2 \mod 65 = 1$
        \item Erstes If: $d=1$ und $dd = 64 \neq 1$ aber $dd = n-1$, also kein return
        \item Zweites If: $b_4 \neq 1$, also $d$ nicht verändern
        \item $dd = d = 1$
        \item Vierter Schleifendurchlauf: $i=3$
        \item $d = 1 \cdot 1 \mod n = 1$
        \item Erstes If: $d=1$ und $dd=1$, also kein return
        \item Zweites If: $b_i \neq 1$. also $d$ nicht verändern.
        \item $dd = d= 1$.
        \item Ab hier wird $d$ und $dd$ nicht mehr verändert, da $d \cdot d = d$ und $b_i = 0\ \forall i \neq 6$.
        \item Also ist das Ergebniss durch $d=1$ gegeben
    \end{enumerate} 
    Der Primzahltest klassifiziert $65$ als Primzahl (obwohl es offensichtlich kein
    Primzahl ist).
    
    \subsection{$a=12$}
    \begin{enumerate}
        \item Schleife: $i=k=6$
        \item $d=1$
        \item Erstes If: $dd=1$, also kein return
        \item Zweites If: $b_i=1$, also $d = d \cdot 12 \mod 65 = 12$
        \item $dd = 12$
        \item Zweiter Schleifendurchlauf: $i=5$
        \item $d=12 \cdot 12 \mod 65 = 14$ 
        \item Erstes If: Kein Return
        \item Zweites If: $d$ nicht verändern
        \item $dd=d=14$
        \item Dritter Schleifendurchlauf: $i=4$
        \item $d=14 \cdot 14 \mod 65 = 1$
        \item $d=1$ und $dd = 14 \neq 1$ und $dd = 14 \neq n - 1 = 64$, also return 0
    \end{enumerate}
    Mit $a=12$ wird die Zahl korrekterweise als nicht-Primzahl klassifiziert.

    \section{Aufgabe}
    Unter den gegeben Bedingungen soll gelten, dass:
    \begin{equation}
        B_n = \{a \in \{1, \ldots, n-1\} | a^{k \cdot 2^i} \equiv \pm 1 \mod n\}
    \end{equation}
    eine echte Untergruppe von $\mathbb{Z}^*_n$ ist.

    Also soll gelten:
    \begin{itemize}
        \item $B_n \subset \mathbb{Z}^*_n$ d.h. $x \in B_n \Rightarrow x \in \mathbb{Z}^*_n \land \exists y \in \mathbb{Z}^*_n$ mit $y \notin B_n$
        \item $B_n$ ist eine Gruppe
    \end{itemize}

    Mit der Definition von $\mathbb{Z}^*_n$ folgt:
    \begin{itemize}
        \item Teilmenge: $\forall b \in B_n: \text{ggT}(b, n) = 1$.
        \item Echte Teilmenge: $\exists y \in \mathbb{Z}^*_n$ mit $y \notin B_n$
        \item Abgeschlossenheit: $\forall x, y \in B_n: x \cdot y \in B_n$
        \item Neutrales Element: $1 \in B_n$
        \item Inverses Element: $\forall x \in B_n: \exists x^{-1} \in B_n$
    \end{itemize}

    \subsection{Teilmenge}
    Es muss gelten:
    \begin{equation}
        \forall b \in B_n: \text{ggT}(b, n) = 1
    \end{equation}
    das heißt es muss gezeigt werden, dass (mit $k$, $i$ wie in Aufgabenstellung):
    \begin{equation}
        b^{k \cdot 2^i} \equiv \pm 1 \mod n \Rightarrow \text{ggT}(b, n) = 1
    \end{equation}
    was äquivalent ist, zu:
    \begin{equation}
        \text{ggT}(b, n) \neq 1 \Rightarrow b^{k \cdot 2^i} \not\equiv \pm 1 \mod n
    \end{equation}

    Es gilt $i < u$. Daraus folgt:
    \begin{equation}
        k \cdot 2^i < k \cdot 2^u 
    \end{equation}
    bzw. für ein $c \geq 1$:
    \begin{equation}
        k \cdot 2^i = \frac{k \cdot 2^u}{2^c}
    \end{equation}

    Für alle Element aus $B_n$ gilt:
    \begin{equation}
        b^{k \cdot 2^i} \equiv \pm 1 \mod n
    \end{equation}
    anders formuliert:
    \begin{eqnarray}
        b^{\frac{k \cdot 2^u}{2^c}} &\equiv& \pm 1 \mod n \\
        \Leftrightarrow b^{k \cdot 2^u} &\equiv& {\left(\pm 1\right)}^{2^c} \mod n
    \end{eqnarray}
    da $c \geq 1$ gilt $2^c$ ist gerade, also gilt:
    \begin{eqnarray}
        b^{k \cdot 2^u} &\equiv& 1 \mod n \\
        b^{n-1} &\equiv& 1 \mod n
    \end{eqnarray}

    Also reicht es zu zeigen, dass:
    \begin{equation}
        \text{ggT}(b, n) \neq 1 \Rightarrow b^{n-1} \not\equiv 1 \mod n
    \end{equation}

    Aus $\text{ggT}(b, n) \neq 1$ folgt:
    \begin{eqnarray}
        b &=& c \cdot f \\
        n &=& m \cdot f \\
    \end{eqnarray}
    mit $\text{ggt}(b, n) = f$. Außerdem gilt: $\text{ggt}(c, m) = 1$, da:
    wäre $\text{ggT}(c, m) > 1$, so wäre $f$ nicht maximal. 
    Da $\tilde{f} = f \cdot \text{ggT}(c, m)$ auch ein Teiler von $n$ und $b$ wäre.
    Zudem gilt $\text{ggT}(b, m) = 1$ und $\text{ggT}(n, c) = 1$, mit gleicher
    Argumentation.

    Außerdem gilt natürlich $c, m, f > 1$.

    Weiteres umformen ergibt:
    \begin{eqnarray}
        b^{n-1} &\not\equiv& 1 \mod n \\
        b^{n} &\not\equiv& b \mod n \\
        {\left(c \cdot f \right)}^{m \cdot f} &\not\equiv& c \dot f \mod m \cdot f \\
        c^{m \cdot f} \cdot f^{m \cdot f} &\not\equiv& c \dot f \mod m \cdot f \\
        c^{m \cdot f} \cdot f^{m \cdot f - 1} &\not\equiv& c \mod m \\
        {\left(c \cdot f\right)}^{m \cdot f -1} &\not\equiv& 1 \mod m \\
        b^{n-1} \equiv n \mod m
    \end{eqnarray}
    Es gilt aber $n = m \cdot f$, daraus folgt:
    \begin{equation}
        n \mod m = 0
    \end{equation}

    Dadurch vereinfacht sich die Gleichung zu:
    \begin{equation}
        b^n \not\equiv 0 \mod m
    \end{equation}

    Das heißt es muss gelten (für $d \in \mathbb{Z}$):
    \begin{equation}
        b^n \neq m \cdot d
    \end{equation}

    Das heißt, es muss gelten:
    \begin{equation}
        \frac{b^n}{m} \neq d
    \end{equation}
    Da $b$ und $m$ Teilerfremd sind ist das erfüllt. 

    \subsection{Echte Teilmenge}
    TODO

    \subsection{Abgeschlossenheit}
    Es muss gelten:
    \begin{equation}
        \forall x, y \in B_n:\ x \cdot y \in B_n
    \end{equation}
    Definiere:
    \begin{equation}
        z = x \cdot y \mod n
    \end{equation}
    Offensichtlich gilt: $z \in \{1, \ldots, n-1\}$, da $x, y \in \{1, \ldots, n-1\}$.

    Außerdem gilt:
    \begin{eqnarray}
        z^{k \cdot 2^i} &=& {\left(x \cdot y\right)}^{k \cdot 2^i} \\
            &=& x^{k \cdot 2^i} \cdot y^{k \cdot 2^i} \\
            &=& (\pm 1) \cdot (\pm 1) \\
            &=& \pm 1
    \end{eqnarray}

    Also gilt: $z = x \cdot y \in B_n$ für beliebige $x, y \in B_n$.

    \subsection{Neutrales Element}
    Es gilt: $1 \in \{1, \ldots, n-1\}$, sowie:
    \begin{equation}
        1^{k \cdot 2^i} = 1 \mod n
    \end{equation}
    also gilt: $1 \in B_n$.

    \subsection{Inverses Element}
    Es muss gelten:
    \begin{equation}
        x \in B_n \Rightarrow x^{-1} \in B_n
    \end{equation}
    es gilt $x^{k \cdot 2^i} = \pm 1$, dann gilt auch:
    \begin{eqnarray}
        {\left(x^{-1}\right)}^{k \cdot 2^i} &=& x^{(-1) \cdot (k \cdot 2^i)} \\
            &=& {\left(x^{k \cdot 2^i}\right)}^{-1} \\
            &=& {\left(\pm 1 \right)}^{-1} \\
            &\stackrel{1^{-1} = 1, {-1}^{-1} = -1}{=} \pm 1
    \end{eqnarray}

    \setcounter{section}{5}
    \section{Aufgabe}
    \pythonexternal{aufgabe6.py} 
\end{document}
