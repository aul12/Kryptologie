\documentclass[DIN, pagenumber=false, fontsize=11pt, parskip=half]{scrartcl}

\usepackage{amsmath}
\usepackage{amsfonts}
\usepackage{amssymb}
\usepackage{enumitem}
\usepackage[utf8]{inputenc} % this is needed for umlauts
\usepackage[ngerman]{babel}
\usepackage[T1]{fontenc} 
\usepackage{commath}
\usepackage{xcolor}
\usepackage{booktabs}
\usepackage{float}
\usepackage{tikz-timing}
\usepackage{tikz}
\usepackage{multirow}
\usepackage{colortbl}
\usepackage{xstring}
\usepackage{circuitikz}
\usepackage{listings} % needed for the inclusion of source code
\usepackage[final]{pdfpages}
\usepackage{subcaption}
\usepackage{import}
\usepackage{cleveref}
\usepackage{bm}

\usetikzlibrary{calc,shapes.multipart,chains,arrows}


% Default fixed font does not support bold face
\DeclareFixedFont{\ttb}{T1}{txtt}{bx}{n}{12} % for bold
\DeclareFixedFont{\ttm}{T1}{txtt}{m}{n}{12}  % for normal

\definecolor{deepblue}{rgb}{0,0,0.5}
\definecolor{deepred}{rgb}{0.6,0,0}
\definecolor{deepgreen}{rgb}{0,0.5,0}

\newcommand\pythonstyle{\lstset{
    language=Python,
    basicstyle=\scriptsize,
    otherkeywords={self},             % Add keywords here
    keywordstyle=\color{deepblue},
    emph={MyClass,__init__},          % Custom highlighting
    emphstyle=\color{deepred},    % Custom highlighting style
    stringstyle=\color{deepgreen},
    frame=tb,                         % Any extra options here
    showstringspaces=false,            % 
}}


% Python environment
\lstnewenvironment{python}[1][]
{
\pythonstyle
\lstset{#1}
}
{}

% Python for external files
\newcommand\pythonexternal[2][]{{
\pythonstyle
\lstinputlisting[#1]{#2}}}

% Python for inline
\newcommand\pythoninline[1]{{\pythonstyle\lstinline!#1!}}

\newcommand{\Prb}[1]{P(\text{#1})}
\newcommand{\CPr}[2]{P(\text{#1}|\text{#2})}
\DeclareMathOperator*{\argmax}{arg\,max}
\DeclareMathOperator*{\argmin}{arg\,min}
\DeclareMathOperator{\rank}{rank}
\newcommand{\R}[0]{\mathbb{R}}

\title{Übungsblatt 3}
\subtitle{Kryptologie: Algorithmen und Methoden}
\author{Paul Nykiel}

\begin{document}
    \maketitle
    \section{Aufgabe}
    Durchschnittlicher Informationsgehalt:
    \begin{equation}
        H(x) = -\sum_x p_x \log_2(p_x) = - \left(0.5 \log_2(0.5) + 0.48 \log_2(0.48) + 0.02 \log_2(0.02)\right) = 1.12
    \end{equation}

    Informationsgehalt \glqq{}Zwilling\grqq{}:
    \begin{equation}
        H(\text{Zwilling}) = -\log_2(p_\text{Zwilling}) = 5.64
    \end{equation}

    \section{Aufgabe}
    Statistische Unabhängigkeit $\Leftrightarrow H(X, Y) = H(X) + H(Y)$.
    
    Nachrechnen: Dafür zuerst die Wahrscheinlichkeiten bestimmen (mithilfe Satz über totale Wahrscheinlichkeit):
    \begin{eqnarray*}
        p_{y_1} &=& p_{y_1, x_1} + p_{y_1, x_2} = \frac{1}{2} \\
        p_{y_2} &=& p_{y_2, x_1} + p_{y_2, x_2} = \frac{3}{8} \\
        p_{y_3} &=& p_{y_3, x_1} + p_{y_3, x_2} = \frac{1}{8} \\
        p_{x_1} &=& p_{y_1, x_1} + p_{y_2, x_1} + p_{y_3, x_1} = \frac{1}{3} \\
        p_{x_2} &=& 1 - p_{x_1} = \frac{2}{3}
    \end{eqnarray*}
    Entropien bestimmen:
    \begin{eqnarray*}
        H(Y) &=& -\sum_y p_{y_k} \log_2(p_{y_k}) = 1.41 \\
        H(X) &=& -\sum_x p_{x_k} \log_2(p_{x_k}) = 0.92 \\
        H(X,Y) &=& -\sum_x \sum_y p_{x_k, y_k} \log_2 (p_{x_k, y_k}) = 2.32
    \end{eqnarray*}

    $\Rightarrow H(X, Y) = H(X) + H(Y) \Rightarrow X$ und $Y$ statistisch unabhängig.

    \section{Aufgabe}
    \begin{enumerate}[label=(\alph*)]
        \item 
            \begin{eqnarray*}
                H(X, Y) &=& -\sum_x \sum_y p_{x_k, y_k} \log_2(p_{x_k, y_k}) \\
                        &=& -\left(\frac{1}{4} \log_2\left(\frac{1}{4}\right) +\frac{1}{4} \log_2\left(\frac{1}{4}\right) +\frac{1}{4} \log_2\left(\frac{1}{4}\right) +\frac{1}{4} \log_2\left(\frac{1}{4}\right) + 0 + 0\right) \\
                        &=& -\log_2\left(\frac{1}{4}\right) \\
                        &=& 2 
            \end{eqnarray*}

            Für $H(X)$ und $H(Y)$ Wahrscheinlichkeiten bestimmen (mithilfe Satz über totale Wahrscheinlichkeit):
            \begin{eqnarray*}
                p_{y_1} &=& p_{y_1, x_1} + p_{y_1, x_2} = \frac{1}{2} \\
                p_{y_2} &=& p_{y_2, x_1} + p_{y_2, x_2} = \frac{1}{4} \\
                p_{y_3} &=& p_{y_3, x_1} + p_{y_3, x_2} = \frac{1}{4} \\
                p_{x_1} &=& p_{y_1, x_1} + p_{y_2, x_1} + p_{y_3, x_1} = \frac{3}{4} \\
                p_{x_2} &=& p_{y_1, x_2} + p_{y_2, x_2} + p_{y_3, x_2} = \frac{1}{4}
            \end{eqnarray*}

            Entropien bestimmen:
            \begin{eqnarray*}
                H(X) &=& -\sum_x p_{x_k} \log_2(p_{x_k}) \\
                    &=& -\left(\frac{3}{4} \log_2\left(\frac{3}{4}\right) + \frac{1}{4} \log_2\left(\frac{1}{4}\right)\right) \\
                    &=& 0.81
            \end{eqnarray*}
            sowie
            \begin{eqnarray*}
                H(Y) &=& -\sum_y p_{y_k} \log_2(p_{y_k}) \\
                    &=& -\left(\frac{1}{2} \log_2\left(\frac{1}{2}\right) +\frac{1}{4} \log_2\left(\frac{1}{4}\right) +\frac{1}{4} \log_2\left(\frac{1}{4}\right)\right) \\
                    &=& 1.5
            \end{eqnarray*}

        \item Es gilt: $H(X|Y) = H(X, Y) - H(Y)$:
            \begin{eqnarray*}
                H(X|Y) &=& H(X, Y) - H(Y) = 0.5 \\
                H(Y|X) &=& H(Y, X) - H(X) = 1.19 \\
            \end{eqnarray*}
        \item
            \begin{equation}
                I(X, Y) = H(X) - H(X|Y) = 0.31
            \end{equation}
    \end{enumerate}

    \section{Aufgabe}
    \begin{enumerate}[label=\alph*)]
        \item Es soll gelten $a>b$, also wird $\text{ggT}(22, 9)$ berechnet:
            \begin{eqnarray*}
                22 &=& 2 \cdot 9\ \text{Rest}\ 4\\ 
                9  &=& 2 \cdot 4\ \text{Rest}\ 1\\
                4  &=& 4 \cdot 1\ \text{Rest}\ 0
            \end{eqnarray*}
            $\Rightarrow \text{ggT}(9, 22) = \text{ggT}(22, 9) = 1$.
        \item Es gilt $e=9, n = 23$, d.h. es muss gelten $9 \cdot d \equiv 1 \mod 22$.
            Sowie $\text{ggT}(9, 22) = 1$, damit lässt sich das multiplikativ
            Inverse mithilfe des $\text{extggT}$ berechnen:
            \begin{equation}
                1 = 9 - 2 \cdot 4 = 9 - 2 \cdot (22 - 2 \cdot 9) = 5 \cdot 9 - 2 \cdot 22
            \end{equation}
            Es gilt also:
            \begin{eqnarray*}
                5 \cdot 9 - 2 \cdot 22 &\equiv& 1 \mod 22 \\
                \Leftrightarrow 5 \cdot 9 &\equiv& 1 \mod 22 \\
                \Leftrightarrow 9 \cdot 5 &\equiv& 1 \mod 22 \\
                \Rightarrow d = 5
            \end{eqnarray*}
        \item Es gilt: $e=9$, $m=16$, $n=23$:
            \begin{equation}
                c = m^e \mod n = 16^9 \mod 23 = 8
            \end{equation}
            Entschlüsseln:
            \begin{equation}
                \hat{m} = c^d \mod n = 8^5 \mod 23 = 16 = m
            \end{equation}
        \item Die Abbildung (siehe Tabelle \ref{tab:ph}) ersetzt die Zeichen (eindeutig)
            durch andere Zeichen ohne ein erkennbares System. Siehe auch \ref{ap:4}.
            \begin{table}[H]
                \centering
                \begin{tabular}{cc}
                    \toprule
                    $m$ & $c$ \\
                    \midrule
                    1 & 1 \\
                    2 & 6 \\
                    3 & 18 \\
                    4 & 13 \\
                    5 & 11 \\
                    6 & 16 \\
                    7 & 15 \\
                    8 & 9 \\
                    9 & 2 \\
                    10 & 20 \\
                    11 & 19 \\
                    12 & 4 \\
                    13 & 3 \\
                    14 & 21 \\
                    15 & 14 \\
                    16 & 8 \\
                    17 & 7 \\
                    18 & 12 \\
                    19 & 10 \\
                    20 & 5 \\
                    21 & 17 \\
                    22 & 22 \\                
                    \bottomrule
                \end{tabular}
                \caption{Abbildung der Pohlig-Hellmann Verschlüsselung}
                \label{tab:ph}
            \end{table}
        \item Es stehen $22! = 1124000727777607680000$ solcher Permutationen zur Verfügung
    \end{enumerate}

    \appendix
    \section{Aufgabe 4} \label{ap:4}
    \pythonexternal{aufgabe4.py}
\end{document}
