\documentclass[DIN, pagenumber=false, fontsize=11pt, parskip=half]{scrartcl}

\usepackage{amsmath}
\usepackage{amsfonts}
\usepackage{amssymb}
\usepackage{enumitem}
\usepackage[utf8]{inputenc} % this is needed for umlauts
\usepackage[ngerman]{babel}
\usepackage[T1]{fontenc} 
\usepackage{commath}
\usepackage{xcolor}
\usepackage{booktabs}
\usepackage{float}
\usepackage{tikz-timing}
\usepackage{tikz}
\usepackage{multirow}
\usepackage{colortbl}
\usepackage{xstring}
\usepackage{circuitikz}
\usepackage{listings} % needed for the inclusion of source code
\usepackage[final]{pdfpages}
\usepackage{subcaption}
\usepackage{import}
\usepackage{cleveref}
\usepackage{bm}

\usetikzlibrary{calc,shapes.multipart,chains,arrows}


% Default fixed font does not support bold face
\DeclareFixedFont{\ttb}{T1}{txtt}{bx}{n}{12} % for bold
\DeclareFixedFont{\ttm}{T1}{txtt}{m}{n}{12}  % for normal

\definecolor{deepblue}{rgb}{0,0,0.5}
\definecolor{deepred}{rgb}{0.6,0,0}
\definecolor{deepgreen}{rgb}{0,0.5,0}

\newcommand\pythonstyle{\lstset{
    language=Python,
    basicstyle=\scriptsize,
    otherkeywords={self},             % Add keywords here
    keywordstyle=\color{deepblue},
    emph={MyClass,__init__},          % Custom highlighting
    emphstyle=\color{deepred},    % Custom highlighting style
    stringstyle=\color{deepgreen},
    frame=tb,                         % Any extra options here
    showstringspaces=false,            % 
}}


% Python environment
\lstnewenvironment{python}[1][]
{
\pythonstyle
\lstset{#1}
}
{}

% Python for external files
\newcommand\pythonexternal[2][]{{
\pythonstyle
\lstinputlisting[#1]{#2}}}

% Python for inline
\newcommand\pythoninline[1]{{\pythonstyle\lstinline!#1!}}

\newcommand{\Prb}[1]{P(\text{#1})}
\newcommand{\CPr}[2]{P(\text{#1}|\text{#2})}
\DeclareMathOperator*{\argmax}{arg\,max}
\DeclareMathOperator*{\argmin}{arg\,min}
\DeclareMathOperator{\rank}{rank}
\newcommand{\R}[0]{\mathbb{R}}

\title{Übungsblatt 2}
\subtitle{Kryptologie: Algorithmen und Methoden}
\author{Paul Nykiel}

\begin{document}
    \maketitle
    \textbf{Hinweis: } Diese Abgabe enthält alle Aufgaben des Übungsblatts, nicht
    nur die Aufgaben die gelöst werden müssen. Ich würde mich um Feedback zu allen
    Aufgaben freuen, kann es allerdings verstehen wenn sie nur Aufgabe 2 und 5 korrigieren
    können.

    \section{Aufgabe}
    Koinzidenzindex:
    \begin{equation}
        \text{IC}(p) = \sum_{i=1}^4 {(p_i)}^2 = {\left(\frac{1}{8}\right)}^2 + {\left(\frac{1}{2}\right)}^2 
            + {\left(\frac{1}{4}\right)}^2 + {\left(\frac{1}{8}\right)}^2 = \frac{11}{32} = 0.34
    \end{equation}

    (Shannon-) Entropie:
    \begin{equation}
        H(p) = - \sum_{i=1}^4 p_i \log_2(p_i) = - \left(-3 \cdot \frac{1}{8}
            - 1 \cdot \frac{1}{2} - 2 \cdot \frac{1}{4} - 3 \cdot \frac{1}{8}\right)
            = \frac{3}{8} + \frac{4}{8} + \frac{4}{8} + \frac{3}{8}
            = \frac{7}{4} = 1.75
    \end{equation}

    \section{Aufgabe}
    \begin{enumerate}[label=\alph*)]
        \item 
            \begin{eqnarray*}
                H(X|Y) &=& \sum_y P(Y=y) H(X|Y=y) \\
                       &=& \sum_y P(Y=y) \sum_x p(X=x|Y=y) \log_2\left(p(X=x|Y=y)\right) \\
                       &=& -\sum_y P(Y=y)\left( \sum_{x} P(X=x|Y=y) \log\left(P(X=x|Y=y)\right)\right) \\
                       &=& -\sum_y P(Y=y) \left(\sum_{x} \frac{P((X=x), Y=y)}{P(Y=y)} 
                                \log\left(\frac{P((X=x), Y=y)}{P(Y=y)}\right) \right)\\
                       &=& -\sum_y P(Y=y) \\
                            &&\left(\sum_{x} \frac{P((X=x), Y=y)}{P(Y=y)} \cdot
                            \left(\log\left(P((X=x), Y=y)\right) -\log\left(P(Y=y)\right)\right)\right)\\
                       &=& -\sum_y P(Y=y) \sum_{x} \frac{P((X=x), Y=y)}{P(Y=y)} \cdot
                            \log\left(P((X=x), Y=y)\right) \\
                            &&- \left(-\sum_{x} \frac{P((X=x), Y=y)}{P(Y=y)} 
                            \cdot \log\left(P(Y=y)\right)\right)\\
                       &=& -\sum_y P(Y=y) \frac{1}{P(Y=y)} \sum_{x} P((X=x), Y=y) \cdot
                            \log\left(P((X=x), Y=y)\right) \\
                            &&- \left(- \frac{\log\left(P(Y=y)\right)}{P(Y=y)} 
                            \sum_{x} P((X=x), Y=y)\right) \\
                       &=& \sum_y P(Y=y)\left(\frac{1}{P(Y=y)} H(X,Y=y) - \left(- \frac{\log\left(P(Y=y)\right)}{P(Y=y)} \cdot P(Y=y) \right)\right)\\
                       &=& \sum_y P(Y=y) \left(\frac{1}{P(Y=y)} H(X,Y=y) - \left(- \log\left(P(Y=y)\right) \right)\right)\\
                       &=& \sum_y P(Y=y) \frac{1}{P(Y=y)} H(X,Y=y) - \left(- \sum_y P(Y=y) \log\left(P(Y=y)\right) \right)\\
                       &=& \sum_y H(X,Y=y) - H(Y)\\
                       &=& H(X,Y) - H(Y)\\
            \end{eqnarray*}
        \item
            Es gilt:
            \begin{equation} \label{eq:HPos}
                H(X) \geq 0 \forall x
            \end{equation}
            da
            \begin{equation} \label{eq:HPosProv}
                P(x) \geq 0 \land P(x) \leq 1 \Rightarrow \log_2(P(x)) \leq 0
            \end{equation} 
            außerdem gilt:
            \begin{equation}
                H(X, Y) = H(Y, X) = H(Y|X) + H(X)
            \end{equation}
            da $H(Y,X) \geq 0$ (\ref{eq:HPos} und \ref{eq:HPosProv}) folgt:
            \begin{equation}
                H(X, Y) \geq H(X)
            \end{equation}
    \end{enumerate}

    \section{Aufgabe}
    \begin{eqnarray*}
        t(n) = n^{\ln(n)} &\geq& 2^{80} \\
        \Leftrightarrow \log_2 \left(n^{\ln(n)}\right) &\geq& 80\\
        \Leftrightarrow \ln(n) \log_2(n) &\geq& 80 \\
        \Leftrightarrow \ln(n) \left(1 + \frac{1}{\log_2(n)} \right) &\geq& 80 \\
        \Leftrightarrow \ln(n) &\geq& \frac{80}{1 + \frac{1}{\log_2(n)}} \\
        \Leftrightarrow n &\geq& \exp\left(\frac{80}{1 + \frac{1}{\log_2(n)}}\right) \\
        \Leftrightarrow n &\geq& 1.67 \cdot 10^{14}
    \end{eqnarray*}

    \section{Aufgabe}
    Für $F_1$ und $F_M$ gilt:
    \begin{itemize}
        \item $F_1 \neq F_M$ für alle Pixel die Teil von \glqq{}Hallo\grqq{} sind
        \item $F_1 = F_M$ für alle anderen Pixel
    \end{itemize}
    Für $F_2$ und $F_M$ gilt:
    \begin{itemize}
        \item $F_2 = F_M$ für alle Pixel die Teil von \glqq{}Welt\grqq{} sind
        \item $F_2 \neq F_M$ sonst
    \end{itemize}
    Daraus folgt (jedes Pixel gehört entweder zu \glqq{}Hallo\grqq{}, zu \glqq{}Welt\grqq{} oder ist Hintergrund):

    Für alle Pixel die zu \glqq{}Hallo\grqq{} gehören:
    \begin{equation*}
        F_1 \neq F_M \neq F_2 \Leftrightarrow F_1 = F_2
    \end{equation*}

    Für alle Pixel die zu \glqq{}Welt\grqq{} gehören:
    \begin{equation*}
        F_1 = F_M = F_2
    \end{equation*}

    Für alle anderen Pixel:
    \begin{equation*}
        F_1 = F_M \neq F_2 \Leftrightarrow F_1 \neq F_2
    \end{equation*}

    Also ist das Bild das aus $F_1$ und $F_2$ resultiert grau an allen Stellen
    an denen \glqq{}Hallo\grqq{} oder \glqq{}Welt\grqq{} war, und schwarz an
    allen anderen Stellen. In kurz: es ist \glqq{}Hallo Welt\grqq{} in grau
    auf schwarzem Hintergrund zu lesen.

    \section{Aufgabe}
    \begin{enumerate}[label=\alph*)]
        \item Siehe Abbildung \ref{fig:shiftReg}.
            \begin{figure}[H]
                \centering
                \begin{tikzpicture}
                    \filldraw [draw=black, fill=gray!50] (1,0) rectangle (3, 2);
                    \filldraw [draw=black, fill=gray!50] (4,0) rectangle (6, 2);
                    \filldraw [draw=black, fill=gray!50] (7,0) rectangle (9, 2);
                    \filldraw [draw=black, fill=gray!50] (10,0) rectangle (12, 2);
            
                    \draw[line width=1pt,->] (3,1) -- (4,1);
                    \draw[line width=1pt,->] (6,1) -- (7,1);
                    \draw[line width=1pt,->] (9,1) -- (10,1);
    
                    \filldraw [draw=black, fill=white] (2,3) circle (0.4) node {\Huge$\oplus$};
                    \filldraw [draw=black, fill=white] (5,3) circle (0.4) node {\Huge$\oplus$};

                    \draw[line width=1pt,->] (11,2) -- (11,3) -- (5.4,3);
                    \draw[line width=1pt,->] (5,2) -- (5,2.6);
                    \draw[line width=1pt,->] (4.6,3) -- (2.4,3);
                    \draw[line width=1pt,->] (2,2) -- (2,2.6);
                    \draw[line width=1pt,->] (1.6,3) -- (0,3) -- (0,1) -- (1,1);
                \end{tikzpicture}
                \caption{Aufbau des Schieberegisters}
                \label{fig:shiftReg}
            \end{figure}
        \item Bei einem Schieberegister ist die maximale Periodenlänge $2^n$,
            hier gilt $n=4$ also ist hier die maximale Periodenlänge $16$.

            Dies kann z.B. erreicht werden, indem das Schieberegister die Werte
            von 0 bis 15 durchzählt (auch wenn das für Kryptographische Anwendungen
            eher suboptimal ist).
        \item Die Periodenlänge ist abhängig von der initialisierung des Schieberegisters,
            die minimale Periodenlänge beträgt $1$, dies ist zum Beispiel der Fall
            wenn das Schieberegister vollständig mit $0$ initialisert wird.

            Die maximale Periode beträgt 7, z.B. mit Startwert $0001$ (siehe \ref{sec:app:5}).
        
    \end{enumerate}

    \section{Aufgabe}
    \pythonexternal{aufgabe6.py}

    \appendix
    \section{Aufgabe 5} \label{sec:app:5}
    \pythonexternal{aufgabe5.py}
\end{document}
