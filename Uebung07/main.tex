\documentclass[DIN, pagenumber=false, fontsize=11pt, parskip=half]{scrartcl}

\usepackage{amsmath}
\usepackage{amsfonts}
\usepackage{amssymb}
\usepackage{enumitem}
\usepackage[utf8]{inputenc} % this is needed for umlauts
\usepackage[ngerman]{babel}
\usepackage[T1]{fontenc} 
\usepackage{commath}
\usepackage{xcolor}
\usepackage{booktabs}
\usepackage{float}
\usepackage{tikz-timing}
\usepackage{tikz}
\usepackage{multirow}
\usepackage{colortbl}
\usepackage{xstring}
\usepackage{circuitikz}
\usepackage{listings} % needed for the inclusion of source code
\usepackage[final]{pdfpages}
\usepackage{subcaption}
\usepackage{import}
\usepackage{cleveref}
\usepackage{bm}
\usepackage{tcolorbox}

\usetikzlibrary{calc,shapes.multipart,chains,arrows}


% Default fixed font does not support bold face
\DeclareFixedFont{\ttb}{T1}{txtt}{bx}{n}{12} % for bold
\DeclareFixedFont{\ttm}{T1}{txtt}{m}{n}{12}  % for normal

\definecolor{deepblue}{rgb}{0,0,0.5}
\definecolor{deepred}{rgb}{0.6,0,0}
\definecolor{deepgreen}{rgb}{0,0.5,0}

\newcommand\pythonstyle{\lstset{
    language=Python,
    basicstyle=\scriptsize,
    otherkeywords={self},             % Add keywords here
    keywordstyle=\color{deepblue},
    emph={MyClass,__init__},          % Custom highlighting
    emphstyle=\color{deepred},    % Custom highlighting style
    stringstyle=\color{deepgreen},
    frame=tb,                         % Any extra options here
    showstringspaces=false,            % 
}}


% Python environment
\lstnewenvironment{python}[1][]
{
\pythonstyle
\lstset{#1}
}
{}

% Python for external files
\newcommand\pythonexternal[2][]{{
\pythonstyle
\lstinputlisting[#1]{#2}}}

% Python for inline
\newcommand\pythoninline[1]{{\pythonstyle\lstinline!#1!}}

\title{Übungsblatt 7}
\subtitle{Kryptologie: Algorithmen und Methoden}
\author{Paul Nykiel}

\begin{document}
    \maketitle
    \section{Aufgabe}
    Es sind bekannt: $n, e_1, e_2, c_1, c_2$.

    Es gilt:
    \begin{eqnarray}
        c_1 &=& m^{e_1} \mod n \\
        c_2 &=& m^{e_2} \mod n \\
        \text{ggT}(e_1, e_2) &=& 1
    \end{eqnarray}
    
    Da $n = p \cdot q$ eindeutig ist, verwenden Alice und Bob das gleiche
    $p$ und $q$ und dementsprechend auch das selbe $\varphi(n)$.
    
    Das heißt es gilt auch:
    \begin{eqnarray}
        e_1 &\in& \mathbb{Z}^*_{\varphi(n)} \\
        e_2 &\in& \mathbb{Z}^*_{\varphi(n)}
    \end{eqnarray}

    \section{Aufgabe}
    \begin{enumerate}[label=(\roman*)]
        \item 
            \textbf{Hinweis 1:} es wird angenommen, dass $p, q > 1$, 
            sonst gilt die Behauptung für $p=1$, $q=3$ 
            (und damit $n=3$, $\varphi(n)=2$, $x=2$ nicht).

        
            \textbf{Hinweis 2:} o.B.d.A. wird der Beweis wird hier nur mod $p$ geführt,
            gilt aber entsprechend auch mod $q$, durch vertauschen von $p$ und $q$.

            In $\mathbb{Z}^*_p$ gilt:
            \begin{equation}
                x^{\varphi(p)} \equiv 1 \mod p
            \end{equation}
            da $p$ eine Primzahl ist gilt: $\varphi(p) = p-1$,
            d.h. es gilt: 
            \begin{equation}
                x^{p-1} \equiv 1 \mod p
            \end{equation}
            da $q$ ungerade ist gilt, dass $q-1$ gerade ist, daher folgt
            durch potenzieren:
            \begin{eqnarray}
                {\left(x^{p-1}\right)}^{\frac{q-1}{2}} &\equiv& 1^{\frac{q-1}{2}} \mod p \\
                x^{\frac{(p-1)(q-1)}{2}} &\equiv& 1 \mod p
            \end{eqnarray}
            $n$ zerfällt in die beiden Primfaktoren $p$ und $q$ demzufolge gilt:
            \begin{equation}
                \varphi(n) = (p-1)(q-1)
            \end{equation}
            also gilt schlussendlich:
            \begin{equation}
                x^{\frac{\varphi(n)}{2}} \equiv 1 \mod p
            \end{equation}
        \item 
            Es gilt:
            \begin{eqnarray}
                x^{\frac{\varphi(n)}{2}} &\equiv& 1 \mod p \\
                x^{\frac{\varphi(n)}{2}} &\equiv& 1 \mod q \\
                x^{\varphi(n)} &\equiv& 1 \mod n
            \end{eqnarray}
            oder anders geschrieben (mit $k_p, k_q, k_n \in \mathbb{Z}$):
            \begin{eqnarray}
                x^{\frac{\varphi(n)}{2}} &=& 1 + k_p \cdot p \label{eq:1} \\
                x^{\frac{\varphi(n)}{2}} &=& 1 + k_q \cdot  q \label{eq:2} \\
                x^{\varphi(n)} &=& 1 + k_n \cdot  n \label{eq:3}
            \end{eqnarray}
            durch multiplizieren von Gleichung \ref{eq:1} und \ref{eq:2} erhält man:
            \begin{eqnarray}
                x^{\frac{\varphi(n)}{2}} \cdot x^{\frac{\varphi(n)}{2}}
                    &=& (1 + k_p \cdot p) \cdot (1 + k_q \cdot q) \\
                \Leftrightarrow x^{\varphi(n)} &=& 1 + k_q \cdot q + k_p \cdot p + k_q \cdot k_p \cdot p \cdot q
            \end{eqnarray}
            gleichsetzten mit \ref{eq:3} ergibt:
            \begin{eqnarray}
                1 + k_n \cdot n &=& 1 + k_q \cdot q + k_p \cdot p + k_q \cdot k_p \cdot p \cdot q \\
                 k_n \cdot n &=& k_q \cdot q + k_p \cdot p + k_q \cdot k_p \cdot p \cdot q
            \end{eqnarray}
            umformen ergibt:
            \begin{eqnarray}
                 k_n \cdot n &=& k_q \cdot q + k_p \cdot p + k_q \cdot k_p \cdot p \cdot q \\
                 k_n &=& \frac{k_q \cdot q + k_p \cdot p}{n} + k_q \cdot k_p \\
            \end{eqnarray}
            da $k_n \in \mathbb{Z}$ und $k_q \cdot k_p \in \mathbb{Z}$ muss auch
            \begin{equation}
                \frac{k_q \cdot q + k_p \cdot p}{n} = c \in \mathbb{Z}
            \end{equation}
            weiteres umformen ergibt:
            \begin{eqnarray}
                k_q \cdot q + k_p \cdot p &=& c \cdot n \\
                k_q \cdot q + k_p \cdot p &=& c \cdot p \cdot q \\
                k_q \cdot q &=& c \cdot p \cdot q - k_p \cdot p \\
                k_q &=& c \cdot p- \frac{k_p \cdot p}{q} \\
            \end{eqnarray}
            da $k_q \in \mathbb{Z}$ und $c \cdot p \in \mathbb{Z}$ muss auch,
            $\frac{k_p \cdot p}{q} \in \mathbb{Z}$. Da aber $\text{ggT}(p, q)=1$
            muss $k_p$ durch $q$ teilbar sein. Es gilt also:
            \begin{equation}
                k_p = q \cdot c_p,\ \ c_p \in \mathbb{Z}
            \end{equation}

            Einsetzen in Gleichung \ref{eq:1} ergibt:
            \begin{equation}
                x^{\frac{\varphi(n)}{2}} = 1 + c_p \cdot q \cdot p = 1 + c_p \cdot n
            \end{equation}
            also gilt:
            \begin{equation}
                x^{\frac{\varphi(n)}{2}} \equiv 1 \mod n \label{eq:b}
            \end{equation}
        \item 
            Es soll gezeigt werden, dass:
            \begin{equation}
                e \cdot d \equiv 1 \mod \frac{\phi(n)}{2} \Rightarrow
                    x^{e \cdot d} \equiv x \mod n
            \end{equation}
            die linke Seite der Implikation kann auch anders geschrieben werden
            (mit $c \in \mathbb{Z}$):
            \begin{equation}
                e \cdot d = 1 + \frac{\varphi(n)}{2} \cdot c
            \end{equation}
            
            daraus folgt (mit $k \in \mathbb{Z}$):
            \begin{eqnarray}
                x^{e \cdot d} &=& x^{1 + \frac{\varphi(n)}{2}} \\
                    &=& x \cdot x^{\frac{\varphi(n)}{2} \cdot c} \\
                    &=& x \cdot {\left(x^{\frac{\varphi(n)}{2}}\right)}^c \\
                    &\stackrel{\ref{eq:b}}{=}& x \cdot {\left(1 + n \cdot k\right)}^c \\
                    &=& x \cdot \sum_{i=0}^c \binom{c}{i} 1^{c-i} {\left(n \cdot k\right)}^i \\
                    &=& x \cdot \left(1 + \sum_{i=1}^c \binom{c}{i} {\left(n \cdot k\right)}^i\right) \\
                    &=& x + x \cdot n \cdot \left(\sum_{i=1}^c \binom{c}{i} n^{i-1} k^i\right)
            \end{eqnarray}
            Also gilt:
            \begin{equation}
                x^{e \cdot d} \equiv x \mod n
            \end{equation}
        \item Wenn $x$ Teilerfremd zu $n$ ist so lässt sich das Multiplikativ Inverse
            zu $e$ (also $d$) sowohl bezüglich $\varphi(n)$ als auch $\frac{\varphi(n)}{2}$
            berechnen und beide sind zum entschlüsseln nutzbar.
    \end{enumerate}

    \section{Aufgabe}
    \begin{enumerate}[label=(\roman*)]
        \item
            \textbf{Hinweis:} o.B.d.A. wird der Beweis wird hier nur mod $p$ geführt,
            gilt aber entsprechend auch mod $q$, durch vertauschen von $p$ und $q$.

            Es gilt (mit $k \in \mathbb{Z}$):

            \begin{equation}
                m = k \cdot \varphi(n)
            \end{equation}
            außerdem gilt:
            \begin{equation}
                \varphi(n) = (p-1) \cdot (q-1)
            \end{equation}
            das heißt es gilt:
            \begin{equation}
                m = k \cdot (p-1) \cdot (q-1)
            \end{equation}
        
            außerdem gilt (da $p$ eine Primzahl ist):
            \begin{equation}
                \varphi(p) = p-1
            \end{equation}
        
            das heißt es gilt:
            \begin{equation}
                m = \varphi(p) \cdot k \cdot (q-1)
            \end{equation}

            damit gilt dann:
            \begin{eqnarray}
                x^{\varphi(p)} &\equiv& 1 \mod p \\
                {\left(x^{\varphi(p)}\right)}^{k \cdot (q-1)}  &\equiv& 1^{k \cdot (q-1)} \mod p \\
                x^{\varphi(p) \cdot k \cdot (q-1)} &\equiv& 1 \mod p \\
                x^m &\equiv& 1 \mod p \label{eq:a}
            \end{eqnarray}
        \item
            \textbf{Hinweis:} o.B.d.A. wird der Beweis wird hier nur mod $p$ geführt,
            gilt aber entsprechend auch mod $q$, durch vertauschen von $p$ und $q$.

            \textbf{Fall 1:} $a$ Teilerfremd zu $n$:
            
            dann gilt nach Gleichung \ref{eq:a}
            \begin{eqnarray}
                a^m \equiv 1 \mod p \\
                \Leftrightarrow a^{m+1} \equiv m \mod p \\
            \end{eqnarray}

            \textbf{Fall 2:} $a$ nicht Teilerfremd zu $n$:
            
            $n$ hat genau zwei Teiler, $p$ und $q$. D.h. $a$ ist entweder ein
            vielfaches von $p$ oder von $q$.

            \textbf{Fall 2.1:} $a$ ist ein vielfaches von $p$:

            D.h. $a$ hat die Darstellung $a=p \cdot c$ mit $c \in \mathbb{Z}$.

            Dann gilt aber:
            \begin{equation}
                a = p \cdot c \equiv 0 \mod p
            \end{equation}
            und folglich auch:
            \begin{equation}
                a^{m+1} \equiv 0 \mod p
            \end{equation}
            Folglich gilt die Aussage auch für diesen Fall

            \textbf{Fall 2.2:} $a$ ist ein vielfaches von $q$:
    
            D.h. $a$ hat die Darstellung $a=q \cdot c$ mit $c \in \mathbb{Z}$.

            Es gilt wieder (wie oben): $m = (p-1) \cdot (q-1) \cdot k$.

            D.h. es gilt:
            \begin{eqnarray}
                a^{m+1} &\equiv& a \cdot a^m \mod p \\
                a^{m+1} &\equiv& a \cdot {\left(q \cdot c\right)}^{(p-1)\cdot(q-1)\cdot k} \mod p \\
                a^{m+1} &\equiv& a \cdot q^{(p-1)\cdot(q-1)\cdot k} \cdot c^{(p-1)\cdot(q-1)\cdot k} \mod p \\
                a^{m+1} &\equiv& a \cdot {\left(q^{(p-1)}\right)}^{\cdot(q-1)\cdot k} \cdot {\left(q^{(p-1)}\right)}^{\cdot(q-1)\cdot k}
            \end{eqnarray}
            Da gilt $q$ Teilerfremd zu $p$ ist gilt:
            \begin{equation}
                 q^{p-1} \equiv 1 \mod p
            \end{equation}

            Außerdem gilt folgende Abschätzung:
            \begin{eqnarray}
                a &<& n \\
                \Leftrightarrow q \cdot c &<& n \\
                \Leftrightarrow c &<& \frac{n}{q} \\
                \Leftrightarrow c &<& p 
            \end{eqnarray}
            d.h. $p \neq c$, damit folgt, dass $c$ Teilerfremd zu $p$ ist,
            da $p$ eine Primzahl ist, also gilt $p \in \mathbb{Z}^*_p$.

            Damit gilt dann auch:
            \begin{equation}
                c^{p-1} = 1
            \end{equation} 

            daraus folgt:
            \begin{eqnarray}
                a^{m+1} &\equiv& a \cdot {\left(q^{(p-1)}\right)}^{\cdot(q-1)\cdot k} \cdot {\left(q^{(p-1)}\right)}^{\cdot(q-1)\cdot k} \\
                a^{m+1} &\equiv& a \cdot 1^{\cdot(q-1)\cdot k} \cdot 1^{\cdot(q-1)\cdot k} \\
                a^{m+1} &\equiv& a \cdot 1 \cdot 1 \\
                a^{m+1} &\equiv& a \label{eq:m1} \\
            \end{eqnarray}
        \item
            Es muss gelten:
            \begin{equation}
                e \cdot d \equiv 1 \mod \varphi(n)
            \end{equation}
            oder auch (mit $k \in \mathbb{Z}$):
            \begin{equation}
                e \cdot d = 1 + \varphi(n) \cdot k
            \end{equation} 
            Das heißt aber es existiert ein $m$ nach obiger definition, sodass:
            \begin{equation}
                e \cdot d = m + 1
            \end{equation}
            nach Gleichung \ref{eq:m1} gilt dann direkt:
            \begin{equation}
                a^{e \cdot d} = a^{m+1} \equiv a \mod p
            \end{equation} 
            sowie
            \begin{equation}
                a^{e \cdot d} = a^{m+1} \equiv a \mod q
            \end{equation} 

            das heißt es muss gelten (mit $k_p, k_q \in \mathbb{Z}$):
            \begin{eqnarray}
                a^{e \cdot d} &=& a + k_p \cdot p \label{eq:rsa} \\
                a^{e \cdot d} &=& a + k_q \cdot q \\
            \end{eqnarray}
            Gleichsetzten ergibt:
            \begin{eqnarray}
                k_p \cdot p + a &=& k_q \cdot q + a\\
                k_p \cdot p &=& k_q \cdot q\\
            \end{eqnarray}
            da $p$ und $q$ Teilerfremd sind müssen jeweils $k_p$ durch $q$ und $k_q$ durch
            $p$ Teilbar sein.

            Das heißt es gilt: $k_p = q \cdot l,\ \ l \in \mathbb{Z}$.
            Aus Gleichung \ref{eq:rsa} gilt dann aber:
            \begin{equation}
                a^{e \cdot d} = a + l \cdot p \cdot q = a + l \cdot n
            \end{equation}
            also gilt:
            \begin{equation}
                a^{e \cdot d} \equiv a \mod n
            \end{equation}
        \item Diese Aufgabe hat gezeigt, dass beliebige Nachrichten $a$ 
            (mit $a<n$) verschlüsselt (und entschlüsselt) werden können, 
            unabhängig davon ob sie Teilerfremd zu $n$ sind.
    \end{enumerate}

    \section{Aufgabe}
    Es gilt:
    \begin{equation}
        x^2 \equiv y^2 \mod n 
    \end{equation}
    oder auch (für ein $k \in \mathbb{Z}$):
    \begin{equation}
        x^2 = y^2 + n \cdot k
    \end{equation}
    umformen ergibt:
    \begin{eqnarray}
        x^2 - y^2 &=& n \cdot k \\
        \Leftrightarrow (x-y) \cdot (x+y) &=& n \cdot k \\
        \Leftrightarrow (x-y) \cdot (x+y) &=& p \cdot q \cdot k \\
        \Leftrightarrow \frac{(x-y) \cdot (x+y)}{p \cdot q} &=& k
    \end{eqnarray}
    das heißt $(x-y) \cdot (x+y)$ muss sich durch $p$ und $q$ teilen lassen.

    Außerdem gilt $0 < x, y < n$, daraus folgt:
    \begin{equation}
        -n < x - y < n
    \end{equation}
    außerdem gilt $x \neq y$, also kann $x-y$ kein vielfaches von $n$ sein und folglich
    nicht durch $n=p \cdot q$ teilbar sein.

    Das heißt $x+y$ muss entweder durch $p$ oder durch $q$ teilbar sein.
    Da $n$ sowohl durch $p$ als auch durch $q$ teilbar ist gilt dann:
    \begin{equation}
        \text{ggT}(x+y, n) = p \text{ bzw. } q
    \end{equation}
    beides sind nicht triviale Teiler von $n$.
    
\end{document}
