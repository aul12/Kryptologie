\documentclass[DIN, pagenumber=false, fontsize=11pt, parskip=half]{scrartcl}

\usepackage{amsmath}
\usepackage{amsfonts}
\usepackage{amssymb}
\usepackage{enumitem}
\usepackage[utf8]{inputenc} % this is needed for umlauts
\usepackage[ngerman]{babel}
\usepackage[T1]{fontenc} 
\usepackage{commath}
\usepackage{xcolor}
\usepackage{booktabs}
\usepackage{float}
\usepackage{tikz-timing}
\usepackage{tikz}
\usepackage{multirow}
\usepackage{colortbl}
\usepackage{xstring}
\usepackage{circuitikz}
\usepackage{listings} % needed for the inclusion of source code
\usepackage[final]{pdfpages}
\usepackage{subcaption}
\usepackage{import}
\usepackage{cleveref}
\usepackage{bm}
\usepackage{tcolorbox}

\usetikzlibrary{calc,shapes.multipart,chains,arrows}


% Default fixed font does not support bold face
\DeclareFixedFont{\ttb}{T1}{txtt}{bx}{n}{12} % for bold
\DeclareFixedFont{\ttm}{T1}{txtt}{m}{n}{12}  % for normal

\definecolor{deepblue}{rgb}{0,0,0.5}
\definecolor{deepred}{rgb}{0.6,0,0}
\definecolor{deepgreen}{rgb}{0,0.5,0}

\newcommand\pythonstyle{\lstset{
    language=Python,
    basicstyle=\scriptsize,
    otherkeywords={self},             % Add keywords here
    keywordstyle=\color{deepblue},
    emph={MyClass,__init__},          % Custom highlighting
    emphstyle=\color{deepred},    % Custom highlighting style
    stringstyle=\color{deepgreen},
    frame=tb,                         % Any extra options here
    showstringspaces=false,            % 
}}


% Python environment
\lstnewenvironment{python}[1][]
{
\pythonstyle
\lstset{#1}
}
{}

% Python for external files
\newcommand\pythonexternal[2][]{{
\pythonstyle
\lstinputlisting[#1]{#2}}}

% Python for inline
\newcommand\pythoninline[1]{{\pythonstyle\lstinline!#1!}}

\title{Übungsblatt 8}
\subtitle{Kryptologie: Algorithmen und Methoden}
\author{Paul Nykiel}

\begin{document}
    \maketitle
    \section{Aufgabe}
    \begin{enumerate}[label=\alph*)]
        \item Siehe Abbildung \ref{fig:g0}, \ref{fig:g1} und \ref{fig:h}.
            \begin{figure}[H]
                \centering
                \includegraphics[width=\textwidth]{g0.pdf}
                \caption{$G_0$} 
                \label{fig:g0}
            \end{figure}
            \begin{figure}[H]
                \centering
                \includegraphics[width=\textwidth]{g1.pdf}
                \caption{$G_1$} 
                \label{fig:g1}
            \end{figure}
            \begin{figure}[H]
                \centering
                \includegraphics[width=\textwidth]{h.pdf}
                \caption{$H$} 
                \label{fig:h}
            \end{figure}
        \item Die Response von $P$ ist die Permutation welche $G_{b=1}$ zu
            $H$ überführt, also $\sigma$.
    \end{enumerate}

    \section{Aufgabe}
    Fiat-Shamir-Protokoll mit $n=77, \tilde{x}=37$.

    Es muss gelten:
    \begin{equation}
        f(z) = f(x \cdot y \mod n) = \tilde{x} \cdot \tilde{y} \mod n = f(x) \cdot f(y)
    \end{equation}
    mit
    \begin{equation}
        f(x) = x^2 \mod n
    \end{equation}

    \begin{enumerate}[label=\alph*)]
        \item 
            \textbf{Annahme:} Für $b=0$ muss der Prover seinen Wert für $y$ offenlegen,
            nicht $z$ es, wird angenommen, dass $y=63$ (statt $z=63$), sonst hält sich
            Peter nicht an das vorgeschriebene Protokoll.

            Es gilt $\tilde{y}=42$, Challenge $b=0$, Response $y=63$.
            $\tilde{y}$ nachrechnen:
            \begin{equation}
                \tilde{y} = f(y) \stackrel{!}{=} 63^2 \mod 77 = 42
            \end{equation}
            somit besteht die Möglichkeit (mit Wahrscheinlichkeit $\frac{1}{2}$), das
            Peter $x$ kennt.
        \item Es gilt $\tilde{y}=15$, Challenge $b=1$, Response $z=64$.
            Nachrechnen:
            \begin{equation}
                f(z) = 64^2 \mod 77 = 15 \stackrel{!}{=} \tilde{x} \cdot \tilde{y} \mod n
                    = 37 \cdot 15 \mod 77 = 16
            \end{equation}
            da $15 \neq 16$ kennt Peter $x$ definiv nicht.
    \end{enumerate}
\end{document}
