\documentclass[DIN, pagenumber=false, fontsize=11pt, parskip=half]{scrartcl}

\usepackage{amsmath}
\usepackage{amsfonts}
\usepackage{amssymb}
\usepackage{enumitem}
\usepackage[utf8]{inputenc} % this is needed for umlauts
\usepackage[ngerman]{babel}
\usepackage[T1]{fontenc} 
\usepackage{commath}
\usepackage{xcolor}
\usepackage{booktabs}
\usepackage{float}
\usepackage{tikz-timing}
\usepackage{tikz}
\usepackage{multirow}
\usepackage{colortbl}
\usepackage{xstring}
\usepackage{circuitikz}
\usepackage{listings} % needed for the inclusion of source code
\usepackage[final]{pdfpages}
\usepackage{subcaption}
\usepackage{import}
\usepackage{cleveref}
\usepackage{bm}

\usetikzlibrary{calc,shapes.multipart,chains,arrows}


% Default fixed font does not support bold face
\DeclareFixedFont{\ttb}{T1}{txtt}{bx}{n}{12} % for bold
\DeclareFixedFont{\ttm}{T1}{txtt}{m}{n}{12}  % for normal

\definecolor{deepblue}{rgb}{0,0,0.5}
\definecolor{deepred}{rgb}{0.6,0,0}
\definecolor{deepgreen}{rgb}{0,0.5,0}

\newcommand\pythonstyle{\lstset{
    language=Python,
    basicstyle=\scriptsize,
    otherkeywords={self},             % Add keywords here
    keywordstyle=\color{deepblue},
    emph={MyClass,__init__},          % Custom highlighting
    emphstyle=\color{deepred},    % Custom highlighting style
    stringstyle=\color{deepgreen},
    frame=tb,                         % Any extra options here
    showstringspaces=false,            % 
}}


% Python environment
\lstnewenvironment{python}[1][]
{
\pythonstyle
\lstset{#1}
}
{}

% Python for external files
\newcommand\pythonexternal[2][]{{
\pythonstyle
\lstinputlisting[#1]{#2}}}

% Python for inline
\newcommand\pythoninline[1]{{\pythonstyle\lstinline!#1!}}

\title{Übungsblatt 5}
\subtitle{Kryptologie: Algorithmen und Methoden}
\author{Paul Nykiel}

\begin{document}
    \maketitle
    \section{Aufgabe}
    $n=55$ zerfällt in zwei Primfaktoren $p=5$ und $q=11$. Es soll die Quadratwurzel
    von $x=1 \mod n$ berechnet werden. Hierfür wird der Chinesische Restsatz angewandt.
    Dafür werden zuerst zwei Quadratwurzeln $w_1$ und $w_2$ berechnet.

    Offensichtlich gilt:
    \begin{eqnarray}
        1^2 &=& 1 \mod 5 \\
        1^2 &=& 1 \mod 11
    \end{eqnarray}
    also gilt $w_1 = w_2 = 1$. Die vier Quadratwurzeln ergeben sich nun durch
    Rücktransformation der Tuple $(1,1), (1,-1), (-1, 1), (-1, -1)$.

    Hierfür $m_1, m_2$ bestimmen:
    \begin{eqnarray}
        m_1 &=& \frac{n_1 \cdot n_2}{n_1} = n_2 = 11\\
        m_2 &=& \frac{n_1 \cdot n_2}{n_2} = n_1 = 5
    \end{eqnarray}
    Nun die multiplikativ Inversen $y_1, y_2$ bestimmen:
    \begin{eqnarray}
        y_1 &=& m_1^{-1} \mod n_1 = \text{modexp}(m_1, n_1 - 2, n_1) = 11^{5-2} \mod 5 = 1\\
        y_2 &=& m_2^{-1} \mod n_2 = \text{modexp}(m_2, n_2 - 2, n_2) = 5^{11-2} \mod 11 = 9
    \end{eqnarray}

    Damit lässt sich dann die Rücktransformation bestimmen, allgemein gilt:
    \begin{equation}
        x = \left( \sum_{i=1}^k a_i \cdot y_i \cdot m_i \right) \mod n
    \end{equation}

    Damit gilt dann für die einzelnen Tuple:
    \begin{eqnarray}
        x_1 &=& 1 \cdot 1 \cdot 11 + 1 \cdot 9 \cdot 5 \mod 55 = 56 \mod 55 = 1 \\
        x_2 &=& 1 \cdot 1 \cdot 11 + -1 \cdot 9 \cdot 5 \mod 55 = -34 \mod 55 = 21 \\
        x_3 &=& -1 \cdot 1 \cdot 11 + 1 \cdot 9 \cdot 5 \mod 55 = 34 \mod 55  = 34\\
        x_4 &=& -1 \cdot 1 \cdot 11 + -1 \cdot 9 \cdot 5 \mod 55 = -56 \mod 55 = 54 \\
    \end{eqnarray}

    Also sind die Quadratwurzeln von $1 (\mod 55)$: $1, 21, 34, 54$.

    \section{Aufgabe}
    Es sind bekannt: $c, d, e, n$. Außerdem gilt:
    \begin{eqnarray}
        c &=& m \cdot a \mod n \label{eq:2c} \\
        d &=& c \cdot b \mod n \label{eq:2d} \\
        e &=& d \cdot a^{-1} \mod n \label{eq:2e} \\
        m &=& e \cdot b^{-1} \mod n \label{eq:2m}
    \end{eqnarray}
    $d^{-1} \mod n$ lässt sich effizienz über den extggT bestimmen, 
    mit Gleichung \ref{eq:2e} gilt dann:
    \begin{eqnarray}
        e &=& d \cdot a^{-1} \mod n \\
        \Leftrightarrow d^{-1} \cdot e &=& d \cdot d^{-1} \cdot a^{-1} \mod n\\
        \Leftrightarrow d^{-1} \cdot e &=& a^{-1} \mod n\\
        \stackrel{(\ref{eq:2c})}{\Rightarrow} c \cdot d^{-1} e &=& m \cdot a \cdot d^{-1} \cdot e \mod n \\
        \Leftrightarrow c \cdot d^{-1} e &=& m \cdot a \cdot a^{-1} \mod n \\
        \Leftrightarrow c \cdot d^{-1} e &=& m \mod n
    \end{eqnarray} 

    \section{Aufgabe}
    Gegeben $n$ und $d$ oder $e$ lässt sich bei PH effizient $e$ bzw. $d$ berechnen,
    wie es auch bei der Schlüsselerzeugung gemacht wird.
    Folglich kann auch jeder Angreifer des Schlüsselpaar aus einem der beiden Schlüssel
    bestimmen.

    \section{Aufgabe}
    Bei der asymterischen Verschlüsselung wird der öffentliche Schlüssel des Wahlleiters
    von allen Teilnehmern zum verschlüsseln genutzt, so dass nur der Wahlleiter die
    Nachrichten entschlüsseln kann.

    Da es allerdings nur vier verschiedene Nachrichten gibt kann ein Angreifer ebenso
    alle möglichen Nachrichten verschlüsseln und muss dann nur noch die Verschlüsselten
    Nachrichten mit den abgefangenen Nachrichten vergleichen.

    \section{Aufgabe}
    \pythonexternal{aufgabe5.py}
\end{document}
