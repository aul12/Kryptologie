\documentclass[DIN, pagenumber=false, fontsize=11pt, parskip=half]{scrartcl}

\usepackage{amsmath}
\usepackage{amsfonts}
\usepackage{amssymb}
\usepackage{enumitem}
\usepackage[utf8]{inputenc} % this is needed for umlauts
\usepackage[ngerman]{babel}
\usepackage[T1]{fontenc} 
\usepackage{commath}
\usepackage{xcolor}
\usepackage{booktabs}
\usepackage{float}
\usepackage{tikz-timing}
\usepackage{tikz}
\usepackage{multirow}
\usepackage{colortbl}
\usepackage{xstring}
\usepackage{circuitikz}
\usepackage{listings} % needed for the inclusion of source code
\usepackage[final]{pdfpages}
\usepackage{subcaption}
\usepackage{import}
\usepackage{cleveref}
\usepackage{bm}

\usetikzlibrary{calc,shapes.multipart,chains,arrows}


% Default fixed font does not support bold face
\DeclareFixedFont{\ttb}{T1}{txtt}{bx}{n}{12} % for bold
\DeclareFixedFont{\ttm}{T1}{txtt}{m}{n}{12}  % for normal

\definecolor{deepblue}{rgb}{0,0,0.5}
\definecolor{deepred}{rgb}{0.6,0,0}
\definecolor{deepgreen}{rgb}{0,0.5,0}

\newcommand\pythonstyle{\lstset{
    language=Python,
    basicstyle=\scriptsize,
    otherkeywords={self},             % Add keywords here
    keywordstyle=\color{deepblue},
    emph={MyClass,__init__},          % Custom highlighting
    emphstyle=\color{deepred},    % Custom highlighting style
    stringstyle=\color{deepgreen},
    frame=tb,                         % Any extra options here
    showstringspaces=false,            % 
}}


% Python environment
\lstnewenvironment{python}[1][]
{
\pythonstyle
\lstset{#1}
}
{}

% Python for external files
\newcommand\pythonexternal[2][]{{
\pythonstyle
\lstinputlisting[#1]{#2}}}

% Python for inline
\newcommand\pythoninline[1]{{\pythonstyle\lstinline!#1!}}

\title{Übungsblatt 5}
\subtitle{Kryptologie: Algorithmen und Methoden}
\author{Paul Nykiel}

\begin{document}
    \maketitle
    \section{Aufgabe}
    Gesucht: $\chi \in \mathbb{Z}_{55}$ mit $\chi^2 \equiv 1\mod 55$.

    \begin{eqnarray}
        \chi_1 &=& \text{modexp}(1, (55 + 1)/4, 55) \\
               &=& 1^{14} \mod 55 \\
               &=& 1 \\
        \chi_2 &=& 55-\chi_1 \\
               &=& 54
    \end{eqnarray}

    \section{Aufgabe}
    Es sind bekannt: $c, d, e, n$. Außerdem gilt:
    \begin{eqnarray}
        c &=& m \cdot a \mod n \label{eq:2c} \\
        d &=& c \cdot b \mod n \label{eq:2d} \\
        e &=& d \cdot a^{-1} \mod n \label{eq:2e} \\
        m &=& e \cdot b^{-1} \mod n \label{eq:2m}
    \end{eqnarray}
    $d^{-1} \mod n$ lässt sich effizienz über den extggT bestimmen, 
    mit Gleichung \ref{eq:2e} gilt dann:
    \begin{eqnarray}
        e &=& d \cdot a^{-1} \mod n \\
        \Leftrightarrow d^{-1} \cdot e &=& d \cdot d^{-1} \cdot a^{-1} \mod n\\
        \Leftrightarrow d^{-1} \cdot e &=& a^{-1} \mod n\\
        \stackrel{(\ref{eq:2c})}{\Rightarrow} c \cdot d^{-1} e &=& m \cdot a \cdot d^{-1} \cdot e \mod n \\
        \Leftrightarrow c \cdot d^{-1} e &=& m \cdot a \cdot a^{-1} \mod n \\
        \Leftrightarrow c \cdot d^{-1} e &=& m \mod n
    \end{eqnarray} 

    \section{Aufgabe}
    Gegeben $n$ und $d$ oder $e$ lässt sich bei PH effizient $e$ bzw. $d$ berechnen,
    wie es auch bei der Schlüsselerzeugung gemacht wird.
    Folglich kann auch jeder Angreifer des Schlüsselpaar aus einem der beiden Schlüssel
    bestimmen.

    \section{Aufgabe}
    Bei der asymterischen Verschlüsselung wird der öffentliche Schlüssel des Wahlleiters
    von allen Teilnehmern zum verschlüsseln genutzt, so dass nur der Wahlleiter die
    Nachrichten entschlüsseln kann.

    Da es allerdings nur vier verschiedene Nachrichten gibt kann ein Angreifer ebenso
    alle möglichen Nachrichten verschlüsseln und muss dann nur noch die Verschlüsselten
    Nachrichten mit den abgefangenen Nachrichten vergleichen.

    \section{Aufgabe}
    \pythonexternal{aufgabe5.py}
\end{document}
